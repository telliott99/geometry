\documentclass[11pt, oneside]{article} 
\usepackage{geometry}
\geometry{letterpaper} 
\usepackage{graphicx}
	
\usepackage{amssymb}
\usepackage{amsmath}
\usepackage{parskip}
\usepackage{color}
\usepackage{hyperref}

\graphicspath{{/Users/telliott/Github-Math/figures/}}
% \begin{center} \includegraphics [scale=0.4] {gauss3.png} \end{center}

\title{Ptolemy's theorem}
\date{}

\begin{document}
\maketitle
\Large

%[my-super-duper-separator]

In a previous \hyperref[sec:Ptolemy]{\textbf{chapter}} we introduced Ptolemy's theorem.
\begin{center} \includegraphics [scale=0.5] {pt1.png} \end{center}

Here we provide two more proofs of this theorem, as examples of wonderful proofs, and then explore some consequences.
\begin{center} \includegraphics [scale=0.4] {Ptolemy3.png} \end{center}

Above is a graphic from wikipedia that shows where we're going in the first proof.  We will form two sets of similar triangles and use our knowledge about corresponding ratios.

\url{https://en.wikipedia.org/wiki/Ptolemy%27s_theorem}

\subsection*{Ptolemy's theorem from similar triangles}

\label{sec:Ptolemy_similar_triangles}

Find $P$ on the diagonal such that $\angle ADB$ is equal to $\angle PDC$.   We are ignoring the central angle $\angle BDP$ for now --- the open magenta dot.

We can show that $\triangle ABD \sim \triangle PCD$ (see the angles marked equal in the diagram --- two equal is enough).  We have the components of $\angle D$ and also $\angle ABD = \angle PCD$ by the inscribed angle theorem.

\begin{center} \includegraphics [scale=0.4] {Ptolemy12.png} \end{center}

The ratios are $AB/BD/AD = PC/CD/PD$, and in particular
\[ \frac{AB}{BD} = \frac{PC}{CD} \]
\[ AB \cdot CD = PC \cdot BD \]
At the same time, $\triangle APD \sim \triangle BCD$ .  Now the extra angle in the center of $\angle D$ counts in both:  $\angle BDC = \angle ADP$.  Also, $\angle DAC = \angle DBC$ by the inscribed angle theorem.

The ratios are $AP/PD/AD = BC/CD/BD$, and in particular
\[ \frac{AP}{AD} = \frac{BC}{BD} \]
\[ BC \cdot AD = AP \cdot BD \]
Combining the two equations
\[ AB \cdot CD + BC \cdot AD = (AP + PC) \cdot BD \]
\[ = AC \cdot BD \]

$\square$

Yiu also proves the converse theorem.

The dots in this proof make it clear which two pairs of triangles are similar, and I've taken care to list the vertices of the triangles in the same order as the sides in each pair of similar triangles, from smallest to largest.

For an alternate notation see \hyperref[sec:Ptolemy_alt]{\textbf{here}}.

\subsection*{proof by switching sides}

\label{sec:Ptolemy_switch_sides}

\emph{Proof}.  (adapted from wikipedia).

\url{https://en.wikipedia.org/wiki/Ptolemy%27s_theorem}

\begin{center} \includegraphics [scale=0.5] {pt2.png} \end{center}

Let the angle $s$ (red dot) be subtended by arc $AB$ and the angle $t$ (black dot) be subtended by arc $CD$.  Then the central $\angle DPC = s + t$ and it has $\sin s + t$.  The other central $\angle APD$ has the same sine, as it is supplementary to $s + t$.

Let the components of the diagonals be $AC = q + s$ and $BD = p + r$.  

\begin{center} \includegraphics [scale=0.5] {pt3.png} \end{center}

Twice the areas of the four small triangles will then be equal to

\[ 2A = (pq + qr + rs + sp) \sin s + t \]

Simple algebra will show that 

\[ (pq + qr + rs + sp) = (p + r)(q + s) \]
\[ = AC \cdot BD \]

The product of the diagonals times the sine of either central angle is equal to twice the area of the quadrilateral.  

We're on to something.  Now, the great idea.  

\begin{center} \includegraphics [scale=0.5] {pt4.png} \end{center}

Move $D$ to $D'$, such that $AD' = CD$ and $CD' = AD.$   We know that the arithmetic works out because if two pairs of arcs span the same whole, then their sum is equal, and so is the sum of the corresponding chords.

$\triangle ACD \cong \triangle ACD'$  by SSS, so they have the same area.  Therefore the area of $ABCD$ is equal to the area of $ABCD'$.

Some of the angles switch with the arcs.  In particular, angle $t$ (black dot) now corresponds to arc $AD'$.  As a result $s + t$ is the measure of the whole angle at vertex $C$.  The whole angle at vertex $A$ is supplementary, and the sine of the whole angle at vertex $A$ is equal to that at $C$.

So twice the area of $\triangle ABD'$ is $AB \cdot AD' \cdot \sin s + t$, and twice that of $\triangle BCD'$ is $BC \cdot CD' \cdot \sin s + t$.  Add these two areas, equate them with the previous result, and factor out the common term $\sin s + t$:

\[ AC \cdot BD = AB \cdot AD' + BC \cdot CD' \]

But $AD' = CD$ and $CD' = AD$ so

\[ AC \cdot BD = AB \cdot CD + BC \cdot AD \]

This is Ptolemy's theorem. 

$\square$

\textbf{corollaries}

Here are just a few of the results that follow from this remarkable theorem.

\subsection*{equilateral triangle}

\begin{center} \includegraphics [scale=0.2] {equi4.png} \end{center}

Inscribe an equilateral triangle in a circle and pick any point on the circle.

\[ qs = ps + rs \]
\[ q = p + r \]

We proved this earlier, without using Ptolemy's theorem, as Van Schooten's theorem.

Here's a different problem from basically the same diagram (Coxeter).

\begin{center} \includegraphics [scale=0.2] {equi5.png} \end{center}
\[ \frac{1}{PA} + \frac{1}{PC} = \frac{1}{PQ} \]

\emph{Proof}.

Going back to similar triangles:
\[ \triangle AQB \sim \triangle PQC, \ \ \ \ \ \ \triangle AQP \sim \triangle BQC \]
We have then:
\[ AQ:BQ:AB = PQ:QC:PC \]
\[PQ:AQ:PA = QC:BQ:BC \]
From the first:
\[ \frac{AB}{PC} = \frac{AQ}{PQ} \]
and then
\[ \frac{PQ}{PA} = \frac{QC}{BC} \]
Hence 
\[ \frac{1}{PC} + \frac{1}{PA} = \frac{AQ}{AB \cdot PQ} + \frac{QC}{BC \cdot PQ} \]
\[ = \frac{1}{PQ} \cdot (\frac{AQ}{AB} + \frac{QC}{BC} ) \]

But of course $AB = BC$ and $AQ + QC = AC = AB = BC$, so finally:
\[ \frac{1}{PC} + \frac{1}{PA} = \frac{1}{PQ} \]

\subsection*{Pythagorean theorem}

Let the quadrilateral be a rectangle.  The the sum of squares of opposing sides is
\[ a^2 + b^2 \]

Triangles made by opposing diagonals are congruent, so the diagonals are equal in length.  The diagonal is the hypotenuse, hence
\[ a^2 + b^2 = c^2 \]

We saw this proof previously (\hyperref[sec:PProof_Ptolemy]{\textbf{here}}).

\subsection*{Law of Cosines}

\label{sec:LOC_by_Ptolemy}

Draw $\triangle ABC$ (suppress the $B$ label) and then draw another triangle congruent with it, with a shared base, and all four points in a circle, forming a cyclic quadrilateral.  

Relying on previous work with a rectangle in a circle (\hyperref[sec:rectangle_side_on_a_circle]{\textbf{here}}), we know this construction is possible.

The points are $A, A', C, C'$.  $\beta$ marks the original $\angle B$, but will not be used.

\begin{center} \includegraphics [scale=0.15] {law_of_cosines3.png} \end{center}

We need an expression for $x$.  We have that the base of the altitude from $C$ to side $c$ is a distance from $A$ equal to $(c-x)/2$.  It follows that
\[ \frac{c-x}{2} \div b = \cos A \]
\[ c - x = 2b \cos A \]
\[ x = c - 2b \cos A \]

Now, apply Ptolemy's Theorem.  We have:
\[ a^2 = b^2 + cx \]
\[ = b^2 + c(c - 2b \cos A) \]
\[ = b^2 + c^2 - 2bc \cos A \]

$\square$

\subsection*{golden mean in the pentagon}

\begin{center} \includegraphics [scale=0.3] {Ptolemy5.png} \end{center}

Take four vertices of the regular pentagon and draw two diagonals.  From the theorem, we have
\[ b \cdot b = a \cdot a + a \cdot b \]
\[ \frac{b^2}{a^2} = 1 + \frac{b}{a} \]

Rather than use the quadratic equation, rearrange and add $1/4$ to both sides to ``complete the square":
\[ \frac{b^2}{a^2} - \frac{b}{a} + \frac{1}{2^2} = 1 + \frac{1}{2^2} \]

So
\[ (\frac{b}{a} - \frac{1}{2})^2  = \frac{5}{4} \]
\[ \frac{b}{a} - \frac{1}{2}  = \pm \ \frac{\sqrt{5}}{2} \]
\[ \frac{b}{a}  = \frac{1 \pm \sqrt{5}}{2} \]

This ratio $b/a$ is known as $\phi$, the golden mean.

\subsection*{diagonals}

Let us look at something like what we used for the proof of Ptolemy's theorem in the beginning.

\begin{center} \includegraphics [scale=0.5] {pt6.png} \end{center}
\[ nm = ac + bd \]

We move one of the points, exchanging sides $a$ and $d$.  Then, one of the diameters, $n$, changes length to $u$.
\[ mu = ab + cd \]

If, instead, we exchange sides $a$ and $b$, the old $m$ changes to $u$.  Why?

\begin{center} \includegraphics [scale=0.5] {pt7.png} \end{center}

Mark the peripheral angles with equal arcs ($abcd$:  red, blue, green, magenta).

The triangle with sides $b$ and $d$ in the middle, and magenta plus red for the vertex angle, is congruent to one in the right panel.  So their long sides are equal, both have length $u$.

Thus,
\[ nu = ad + bc \]

We get a formula for the square of the diagonal:
\[ m^2 = \frac{(mu)(nm)}{nu} = \frac{(ab + cd)(ac + bd)}{(ad + bc)}  \]

There is a similar formula for $n^2$.  These formulas are sometimes attributed to Brahmagupta.  This beautiful proof is due to Paramesvara (14th century).

\url{https://www.cut-the-knot.org/proofs/PtolemyDiagonals.shtml}

The ratio is
\[ \frac{m}{n} = \frac{ab + cd}{ad + bc} \]

which is referred to as Ptolemy's second theorem.

\end{document}