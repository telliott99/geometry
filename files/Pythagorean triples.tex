\documentclass[11pt, oneside]{article} 
\usepackage{geometry}
\geometry{letterpaper} 
\usepackage{graphicx}
	
\usepackage{amssymb}
\usepackage{amsmath}
\usepackage{parskip}
\usepackage{color}
\usepackage{hyperref}

\graphicspath{{/Users/telliott/Dropbox/Github-math/figures/}}
% \begin{center} \includegraphics [scale=0.4] {gauss3.png} \end{center}

\title{Pythagorean triples}
\date{}

\begin{document}
\maketitle
\Large

%[my-super-duper-separator]

The simplest right triangle with integer sides is a $3,4,5$ right triangle:
\[ 3^2 + 4^2 = 5^2 \]

but of course any multiple $k$ will work
\[ (3k)^2 + (4k)^2 = (5k)^2 \]
However, that's not so interesting.  

The triples which are not multiples of another triple are called \emph{primitive}.  There is a small table of triples in this discussion of Euclid X:29 by Joyce:

\url{https://mathcs.clarku.edu/~djoyce/elements/bookX/propX29.html}

\begin{center} \includegraphics [scale=0.4] {triples_joyce.png} \end{center}

We can see that the entries in each column are similar.  For example, in  the first column
\[ (3,4,5) \ \ \ (5,12,13) \ \ \ (7,24,25) \ \ \ (9,40,41) \]

The values $a_n$ differ by a constant:  In column 1, $\Delta$ is $2$:
\[ a_{n+1} = a_n + 2 \]

In the second and third columns, the $\Delta$ for $a_n$ is $6$ and $10$, respectively.

For $b$ and $c$ the $\Delta$ at each step is the same for $b$ and $c$, but the scale ratchets upward.  For the first column, $c = b + 1$ and the step rule is
\[ \Delta = 4 + 4k \]
You can think of the first entry for $b$ as being generated by $k=0$.

The first difference is $8$, then $12$, then $16$ and so on.

It is conventional to write $a$ as odd.  We observe that, in the table, $b$ is even and $c$ is odd.

\subsection*{the basic rule}

Euclid had a formula for triples:
\[ b = 2mn, \ \ \ \ \ \ a = n^2 - m^2, \ \ \ \ \ \ c = n^2 + m^2 \]

\url{https://en.wikipedia.org/wiki/Pythagorean_triple}

Using the formula, we look at some examples:

\begin{verbatim}
m  n   b   a   c
1  2   4   3   5
2  3  12   5  13
3  4  24   7  25
4  5  40   9  41
5  6  60  11  61

1  3   6   8  10
1  4   8  15  17
1  5  10  24  26
1  6  12  35  37

2  4  16  12  20
2  5  20  21  29
2  6  24  32  40

3  5  30  16  34
3  6  36  27  45

4  6  48  20  52
\end{verbatim}

Notice that when the sum of $m$ and $n$ is even, the triple is not a primitive one.  Another one that we don't want is constructed from $m = 3, n = 6$, where the factor of $3$ is obvious.  It seems that $m$ and $n$ should be co-prime, they should have no common factors (other than $1$).

\subsection*{factors}

The elementary rule about squares is this:  if $n$ is even then so is $n^2$, while if $n$ is odd then so is $n^2$.  

To see this, write $n = 2k$ for $k \in 1,2,3 \dots$ as the definition of an even number.  Then $n^2 = 4k^2$, which is even.  

On the other hand, if $n$ is odd, write $n = 2k + 1$ with $k \in 0, 1, 2 \dots$, so $n^2 = 4k^2 + 4k + 1$, which is odd.  Since there are only these two cases, we can conclude that the converse is also true:  an even square comes from an even number and an odd square from an odd number.

As a result, we find that for the triples we care about, $a$ and $b$ are not both even, because then $a^2$ and $b^2$ would be even, and so would $c^2$.  So then $c$ would be even, and the triple would not be primitive.

\subsection*{even and odd}

Let us go back to 
\[ a^2 + b^2 = c^2 \]

We said that $a$ and $b$ cannot both be even, because then $c$ would be even.  Or rather, they can, but in that case we are not interested.

The other possible cases are, either $a$ and $b$ both odd, or one is even and one odd.  In the first case we have that $c$ is even because odd plus odd is even.  So for that case, $a$ and $b$ both odd and $c$ even:

\[ (2i + 1)^2 + (2j + 1)^2 = (2k)^2 \]
\[ 4i^2 + 4i + 4j^2 + 4j + 2 = 4k^2 \]

The left-hand side is not evenly divisible by $4$, but the right-hand side is.  This is impossible.  Hence one of $a$ and $b$ is even and one odd.  As they say, \emph{without loss of generality}, let $a$ be odd, as we saw in the table above.

\subsection*{more about factoring}

Rearrange the equation:
\[ b^2 = c^2 - a^2  = (c + a)(c - a) \]

Since $b$ is even, we can write $b = 2t$
\[ 4t^2 = (c + a)(c - a) \]

Now we come to an argument about common factors.  There are some basic facts we should obtain first.  Let
\[ p + q = r \]
Now, suppose that $p$ and $q$ share a common factor, $f$.  So then
\[ f \cdot i + f \cdot j = f \cdot (i + j) = r \] 

By the fundamental theorem of arithmetic, if $f$ is a factor of the left-hand side, it is also a factor of $r$.  

In a similar way, suppose that $p$ and $r$ share a common factor, $f$.  Then
\[ r - p = f \cdot k - f \cdot i = f \cdot (k-i) = q \] 
and again, all three must have the common factor.  But we have agreed that these cases do not interest us.

The same argument applies to squares, since if there is a common factor, it will be there as $f^2$ when it is present.

We conclude that $a$, $b$ and $c$ must be all relatively prime.  No two of them can share a common factor.

Let us now go back to 
\[ 4t^2 = (c + a)(c - a) \]
\[ t^2 = \frac{(c + a)}{2} \cdot \frac{(c - a)}{2} \]

Recall that $a$ and $c$ are both odd, so their sum and difference are both even.  Therefore
the two factors on the right-hand side are integers, while $t^2$ is a perfect square, namely, that of $t$.  

The sum and difference for these terms are:

\[ (c + a)/2 + (c - a)/2 = c \]
\[ (c + a)/2 - (c - a)/2 = a \]

On the supposition that $(c+a)/2$ and $(c-a)/2$ did have a common factor, then they would share that common factor with both $c$ and $a$.   Since we know that $a$ and $c$ (at least the particular ones we're interested in) do not have a common factor, neither do these two terms.

So the two terms have no common factor and yet multiply together to give a perfect square.

So then
\[ t^2 = xy \]
where $x$ and $y$ are co-prime. 

Suppose that $x$ and $y$ are not perfect squares.  Then
\[ t^2 = (x_1 \cdot x_2 \dots) (y_1 \cdot y_2 \cdots) \]

But this is impossible.  Each factor of $t^2$ must be present an even number of times on the right-hand side, since $t^2$ is a perfect square.  Since no factors are shared between $x$ and $y$, it must at least be that
\[ t^2 = x_1^2 \cdot y_1^2 \]

but possibly
\[ t^2 = x_1^2 \cdot x_2^2 \dots y_1^2 \cdot y_2^2 \]

Therefore both factors are themselves perfect squares.  

That is, there exist integers $m$ and $n$ such that
\[ m^2 =  \frac{(c - a)}{2} \]
\[ n^2 =  \frac{(c + a)}{2} \]
with $n > m$.

Adding
\[ m^2 + n^2 = c \]
Subtracting
\[ n^2 - m^2 = a \]

Go back again to
\[ 4t^2 = (c + a)(c - a) \]
\[ = m^2 n^2 \]
\[ 2t = mn = b \]

We have not limited $m$ and $n$ in any way except to say that they are not equal so one is larger than the other and arbitrarily suppose $n > m$.  Every primitive triple must have an integer $m$ and $n$ with these properties:

\[ c = m^2 + n^2, \ \ \ \ \ a = n^2 - m^2, \ \ \ \ \ \ b^2 = 2mn \]

So finally not only do $m$ and $n$ exist with these properties for any triple, but any integer $m$ and $n$ will satisfy the Pythagorean condition, since:

\[ a^2 + b^2 = (n^2 - m^2)^2 + (2mn)^2 \]
\[ = n^4 - 2n^2m^2 + m^4 + 4n^2m^2 \]
\[ = n^4 + 2n^2m^2 + m^4 \]
\[ = (n^2 + m^2)^2 = c^2 \]

Any integer $m$, $n$, with $n > m$ will work.

For 3-4-5, $n = 2$, $m=1$.

This is a proof that this formula gives all Pythagorean triples.

$\square$

\subsection*{another derivation}

Start with our favorite:
\[ \sin^2 x + \cos^2 x = 1 \]
\[ \tan^2 x + 1 = \frac{1}{\cos^2 x} \]
\[ \cos^2 x = \frac{1}{1 + \tan^2 x} \]

The double-angle formula for sine:
\[ \sin 2s = 2 \sin s \cos s \]
\[ = 2 \frac{\sin s}{\cos s} \cos^2 s \]
\[ = 2 \tan s \ \frac{1}{1 + \tan^2 s} \]

Let $a = \tan s$, then
\[ \sin 2s = \frac{2a}{1 + a^2} \]

\subsection*{cosine}

\[ \cos 2s = \cos^2 s - \sin^2 s \]
\[ = \ [ \ \frac{\cos^2 s}{\cos^2 s} - \frac{\sin^2 s}{\cos^2 s} \ ] \ \cos^2 s \]
\[ = \ [ \ \frac{1 - \tan^2 s}{1 + \tan^2 s} \ ] \]
 
so
\[ \cos 2s = \frac{1 - a^2}{1 + a^2} \]

In general, $a$ can be anything.  

But if $a$ is a rational number, then we can obtain the corresponding sides of a right triangle with rational lengths as well.  

The sides are:  $2a, 1 - a^2$ with the hypotenuse:
\[ \sqrt{4a^2 + (1 - 2a^2 + a^4)} \]
\[ \sqrt{1 + 2a^2 + a^4)} \]
\[ = 1 + a^2 \]

Suppose $a = \frac{2}{3}$.  Then, we have side lengths:  $\frac{4}{3} = \frac{12}{9},\frac{5}{9}$, and $\frac{13}{9}$, which can be converted to integers:  $12, 5, 13$.

In general, if $a = \tan s = p/q$ then the sides are
\[ \frac{2p}{q}, \ \ \ \ \ \ 1 - \frac{p^2}{q^2}, \ \ \ \ \ \ 1 +\frac{p^2}{q^2} \]
which as integers will be
\[ 2pq, \ \ \ \ \ \ q^2 - p^2, \ \ \ \ \ \ q^2 + p^2 \]

\subsection*{examples}

Make a series with $m + 1 = n$:
\[ b = 2mn \]
\[ a = n^2 - m^2 = 2m + 1 \]
\[ c = n^2 + m^2 \]
So then $m = 1,2,3 \dots$:
\[ (3,4,5) \ \ \ (5,12,13) \ \ \ (7,24,25)  \]


\end{document}