\documentclass[11pt, oneside]{article} 
\usepackage{geometry}
\geometry{letterpaper} 
\usepackage{graphicx}
	
\usepackage{amssymb}
\usepackage{amsmath}
\usepackage{parskip}
\usepackage{color}
\usepackage{hyperref}

\graphicspath{{/Users/telliott/Dropbox/Github-math/figures/}}
% \begin{center} \includegraphics [scale=0.4] {gauss3.png} \end{center}

\title{Computations}
\date{}

\begin{document}
\maketitle
\Large

%[my-super-duper-separator]

In this chapter we will continue with the theme of the \hyperref[sec:law_of_cosines]{\textbf{Law of cosines}}.  There is a general theorem and two specific cases.  I think it's perhaps easier if we start with the general case.

\subsection*{Stewart's theorem}

\label{sec:Stewarts_theorem}

\begin{center} \includegraphics [scale=0.5] {bisector3b.png} \end{center}

We have a triangle with sides $a$, $b$ and $c$ and a line segment connecting one vertex to the opposite side, dividing it into segments $x$ and $y$.  In the general case these two segments are not equal ($x \ne y$), and the vertex angle is not bisected ($\angle s \ne \angle t$).  The two angles at the base are $\theta$ and $\theta'$.  We use the law of cosines to write:

\emph{Proof}.

\[ a^2 = e^2 + x^2 - 2ex \cos \theta \]
\[ b^2 = e^2 + y^2 - 2ey \cos \theta' \]

Since $\theta$ and $\theta'$ are supplementary angles, $\cos \theta = - \cos \theta'$ (see \hyperref[sec:signed_angles]{\textbf{here}}), so
\[ b^2 = e^2 + y^2 + 2ey \cos \theta \]

We divide the first equation by $x$ and the second by $y$ and add which makes the term $2e \cos \theta$ disappear:
\[ \frac{a^2}{x} + \frac{b^2}{y} = \frac{e^2 + x^2}{x} + \frac{e^2 + y^2}{y} \]

Get rid of the fractions:
\[ a^2y + b^2x = e^2 (x+y) + x^2y + xy^2 \]
\[ = (e^2 + xy) (x + y) \]

$\square$

This is called Stewart's theorem.

\subsection*{Apollonius' theorem}

\label{sec:Apollonius_theorem}

The first special case is where the line segment divides the base in half, so that $x = y$.  This is called Apollonius' theorem.  

\emph{Proof}.

We had

\[ a^2y + b^2x = (e^2 + xy) (x + y) \]

Substituting $m = x = y$
\[ a^2m + b^2m = (e^2 + mn)(2m) \]
\[ a^2 + b^2 = 2(e^2 + m^2) \]
\[ \frac{a^2 + b^2}{2} = e^2 + m^2 \]

$\square$

The average of the squares of the sides is equal to the median squared plus the half-base squared.

Put another way, if we imagined that the Pythagorean theorem held (it doesn't, in the general case there is not a right angle), but if it did, then this would just be the sum of $a^2 = e^2 + m^2$ and $b^2 = e^2 + m^2$.

\begin{center} \includegraphics [scale=0.5] {bisector3d.png} \end{center}

\subsection*{review of angle bisector theorem}

Let's go back to the generalized case of the angle bisector theorem, which we looked at  (\hyperref[sec:generalized_angle_bisector_theorem]{\textbf{here}}).

\begin{center} \includegraphics [scale=0.5] {bisector1.png} \end{center}

\[ \frac{x}{a} = \frac{y}{b} \]

As we turn the base $c$, but keep the angle bisection relationship, the ratio stays the same.

\emph{Proof}.

One proof starts by recall the \hyperref[sec:area_ratio_theorem]{\textbf{area-ratio theorem}}.  The area of the two small triangles is in the same ratio as the lengths of the bases $x$ and $y$.  On the other hand, we can compute the ratio of (twice) the areas as
\[ \frac{ae \cos s}{be \cos s} = \frac{a}{b} = \frac{x}{y} \]

For a different proof, compute the areas of the two smaller triangles in a different way.  We will form the ratio of the areas using the same component altitudes.

In the figure below, the altitude $g$ can be computed as $g = a \sin s$.  Twice the area of the top triangle is thus $ge = ae \sin s$ (where $e$ is the length of the bisector).  Twice the area of the bottom one is $b$ times the same factor.  So the \emph{ratio} of areas is $a/b$.

\begin{center} \includegraphics [scale=0.5] {bisector3c.png} \end{center}

But the same altitudes can also be computed in terms of the sides $x$ and $y$.  We have that $e = x \sin \theta$ and then twice the area of the top triangle is $ge = gx \sin \theta$.  

$h$ can be computed similarly as $h = y \sin \theta' = y \sin \theta$, since $\theta$ and $\theta'$ are supplementary.  So twice the area of the bottom triangle is $y$ times the same factor.  The ratio of areas is $x/y$.

Since the ratio must be equal no matter how it is calculated we have that
\[ \frac{x}{y} = \frac{a}{b} \]

$\square$

A related proof relies on the law of sines.  We have that 
\[ \frac{x}{\sin s} = \frac{a}{\sin \theta} \]
\[ \frac{y}{\sin t} = \frac{b}{\sin \theta'} \] 

\begin{center} \includegraphics [scale=0.5] {bisector3b.png} \end{center}

But $s = t$ so $\sin s = \sin t$ and also $\theta$ and $\theta'$ are supplementary so $\sin \theta = \sin \theta'$.  We obtain
\[ \frac{x}{a} = \frac{\sin s}{\sin \theta} = \frac{\sin t}{\sin \theta'} = \frac{y}{b} \]

$\square$

\subsection*{bisected angle and Stewart's theorem}

\label{sec:Stewart_bisected}

We use the result for the bisected angle to simplify Stewart's theorem for that case.

\begin{center} \includegraphics [scale=0.5] {bisector1.png} \end{center}

\emph{Proof}.

For a bisected angle, the basic relationship is 
\[ \frac{x}{a} = \frac{y}{b} \]

which can be rewritten in two ways as
\[ x = \frac{ay}{b} \]
\[ y = \frac{bx}{a} \]

and we can use this to simplify Stewart's theorem for the special case. 
\[ a^2y + b^2x = (e^2 + xy) \cdot (x + y) \]

The substitute separately for $y$ and $x$ on the left-hand side:
\[ a^2y + b^2x = aby + abx = ba(x + y) \]
That cancels the second term on the right-hand side of Stewart's theorem, leaving:
\[ ab = e^2 + xy \]

$\square$

\subsection*{median in terms of side lengths}

This problem is Hopkins 966.

\begin{center} \includegraphics [scale=0.4] {Hopkins_966.png} \end{center}

We just use Apollonius' theorem.  
\[ a^2 + b^2 = 2(e^2 + m^2) \]

We must change symbols.  We had $e$ for the line segment dropping from the vertex, but the problem uses $m$ for $e$ (because it's a median).  If $c$ is the length of the base, then $c/2$ is the half-length, and we have

\[ a^2 + b^2 = 2(m^2 + (c/2)^2) \]
Lose the fraction for a bit:
\[ 2(a^2 + b^2) = 4m^2 + c^2 \]

\[ 4m^2 = 2(a^2 + b^2) - c^2 \]
\[ m = \frac{1}{2} \ \sqrt{2(a^2 + b^2) - c^2} \]

\subsection*{Hopkins 967}

\begin{center} \includegraphics [scale=0.5] {Hopkins_967.png} \end{center}

I thought, how hard can this be?  If you know Stewart's theorem, it's not hard.

First of all, we notice that $\triangle ACD \sim \triangle BDE$, because $\angle A = \angle E$ (equal arcs), and there is a shared vertical angle.  That helps only in the sense that it reminds us of the main result about angle bisectors that we revisited above.

Their solution starts with the statement.

\[ k^2 = ab - AD \cdot DB \]

(switching from $e$ in our notation to $k$).  I much prefer to use $x$ for $BD$ and $y$ for $AD$, which gives
\[ k^2 = ab - xy \]

By this time we recognize this as the simplified form of Stewart's theorem for the special case of the bisected angle.  Leave that for the moment.  

Recall the result from the angle bisector theorem and rewrite it slightly
\[ \frac{x}{y} = \frac{a}{b} \]
Add one to both sides
\[ \frac{x + y}{y} = \frac{a + b}{b} \]
\[ \frac{c}{y} = \frac{a + b}{b} \]
\[ \frac{c}{a + b} = \frac{y}{b} = \frac{x}{a} \]

This manipulation was made famous by Archimedes and he used it in his work on the estimation of $\pi$.  

Back to our problem:
\[ y = \frac{bc}{a + b} \]
\[ x = \frac{ac}{a + b} \]

Finally, we can use the original expression to write:
\[ k^2 = ab -  \frac{bc}{a + b} \cdot \frac{ac}{a + b} \]

which can be manipulated a bit  
\[ = ab \ [1 - \frac{c^2}{(a + b)^2}] \]

We have an expression for the length of the angle bisector in terms of the sides of the triangle.  It is symmetric in $a$ and $b$, which we should expect.

\subsection*{Archimedes}

Back to what we were saying about Archimedes, briefly.  We had
\[ \frac{c}{a + b} = \frac{y}{b} = \frac{x}{a} \]

Inverted, this equation says that 
\[ \frac{a}{c} + \frac{b}{c} = \frac{b}{y} \]

We can apply this to the case where the triangle is a right triangle.  In trigonometric jargon, it says that the co-tangent (inverted tangent) of the half-angle is equal to the sum of the cotangent of the whole angle plus the cosecant (inverted sine) of the whole angle.

It gives a way of computing the cotangent of the half angle knowing values for the whole angle, and then the Pythagorean theorem gives the cosecant of the half angle since 
\[ \sin^2 \theta + \cos^2 \theta = 1 \]
Divide both sides by $\sin^2 \theta$:
\[ 1 + \cot^2 \theta = \csc^2 \theta \]
\[ \csc \theta = \sqrt{1 + \cot^2 \theta} \]

The perimeter of a polygon of $n$ sides inscribed in a circle of diameter equal to $1$ is $n \sin \phi$, and for a circumscribed polygon it is $n \tan \phi$, where $\phi = \pi/n$.

Archimedes started with a 30-60-90 triangle in a hexagon and carried out the ``side doubling" four times.  He obtained $223/71 < \pi < 22/7 $.

\end{document}