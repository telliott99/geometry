\documentclass[11pt, oneside]{article} 
\usepackage{geometry}
\geometry{letterpaper} 
\usepackage{graphicx}
	
\usepackage{amssymb}
\usepackage{amsmath}
\usepackage{parskip}
\usepackage{color}
\usepackage{hyperref}

\graphicspath{{/Users/telliott/Github-Math/figures/}}

\title{Similar triangles}
\date{}

\begin{document}
\maketitle
\Large

%[my-super-duper-separator]

\subsection*{Euclid}

We repeat the same proof, with more labels, in the style of Euclid.

\label{sec:Euclid_VI_2}

\begin{center} \includegraphics [scale=0.6] {Euclid_VI_2.png} \end{center}

In $\triangle ABC$, let $DE$ be drawn parallel to $BC$, with $AD$ in some ratio to $DB$, call it $k$.

Then, we claim that $AE$ is to $EC$ in the same ratio.

\emph{Proof.}

First, $\triangle BDE$ and $\triangle ADE$ have the same vertex $E$ and their bases, $AD$ and $DB$, lie on the same line, so they have the same altitude.  

\begin{center} \includegraphics [scale=0.4] {Euclid_VI_3a.png} \end{center}

Therefore, by the area-ratio theorem, their areas are in proportion to the lengths of the bases.

If we designate area as $\Delta$:
\[ \frac{\Delta_{ADE}}{\Delta_{BDE}} = \frac{AD}{BD} \]

Similarly, $\triangle CDE$ and $\triangle ADE$ have the same vertex $D$ and their bases, $AE$ and $CE$, lie on the same line (below).  Therefore they have the same altitude and their areas are in proportion to the lengths of the bases.  

\begin{center} \includegraphics [scale=0.4] {Euclid_VI_3b.png} \end{center}
\[ \frac{\Delta_{ADE}}{\Delta_{CDE}} = \frac{AE}{CE} \]

Combining these two results we have that
\[ \frac{\Delta_{CDE}}{\Delta_{BDE}} = \frac{AD}{BD} \cdot \frac{CE}{AE}  \]

Now we use the information about $DE \parallel BC$.

Triangles $\triangle BDE$ and $\triangle CDE$ have the same base, $DE$.

Because the points $B$ and $C$ lie on a line parallel to the base, their altitudes to that base have the same length, and therefore the triangles have the same area.

\begin{center} \includegraphics [scale=0.4] {Euclid_VI_3c.png} \end{center}

This means that in the previous expression, the left-hand side is equal to $1$, leaving
\[ \frac{AE}{CE} = \frac{AD}{BD} \]

$\square$

\subsection*{converse}

\label{sec:Euclid_VI_2_converse}

Euclid also runs this argument backward to prove the converse:  given equal ratios, it follows that $DE$ is parallel to $BC$.  

\emph{Proof}.

Given

\[ \frac{AD}{AB} = \frac{AE}{AC} = r \]

The ratio of areas of two triangles with the same altitude is in proportion to the bases.  Let that ratio be $r$.

\[ \frac{\Delta_{ADE}}{\Delta_{ABE}} = r = \frac{\Delta_{ADE}}{\Delta_{ACD}} \]

It follows that 
\[ \Delta_{ABE} = \Delta_{ACD} \]

and since they have a shared region $\triangle ADE$, the remainder of each one has the same property:
\[ \Delta_{BDE} = \Delta_{CDE} \]

Triangles with the same area and the same base, and the vertex on the same side of the base, have their top vertices lying along a line parallel to the base, so they have the same height.

\[ DE \parallel BC \]

$\square$

Two other proofs relating similarity and equal ratios are given in the appendix.  One relies on a subtle idea from calculus called a limit:  \hyperref[sec:similarity_theorem]{\textbf{AAA similarity theorem}} (Kiselev).  

We also showed that this follows for right triangles easily from the properties of such triangles (\hyperref[sec:similar_right_triangles]{\textbf{here}}), and is related to the area-ratio theorem.  The theorems about right triangles can be used to build a general theory but we have relegated such efforts to an appendix.

Even more proofs are \hyperref[sec:SAS_similar_cosine]{\textbf{here}}.

\subsection*{scaled triangle proof of Pythagoras}

\label{sec:Pythagoras_scaled_triangles}

\emph{Proof.}

Draw a right triangle and label sides $a,b$ and $c$.

\begin{center} \includegraphics [scale=0.5] {pyth1.png} \end{center}
Now, flip and rotate the triangle.  Make a copy of that one and rotate the copy so that the shortest side is to the left.  Enlarge the copy until the two adjacent sides are equal in length.
\begin{center} \includegraphics [scale=0.5] {pyth2.png} \end{center}

These are similar triangles so the angles are all equal, but the sides are proportional, multiplied by a common factor, which here is $b$.

The adjacent sides show that $ab = b$, so this choice of scaling defines $a = 1$.  And since $a = 1$, we can also view each side of the original triangle as being scaled by a factor of $a$.  

We just tried scaling by a factor of $b$ and then by $a$, so that naturally suggests we try scaling by a factor of $c$.  

\begin{center} \includegraphics [scale=0.5] {pyth2b.png} \end{center}
All sides of the enlarged triangle have been multiplied by the same factor, namely, $c$.  Notice that $c = ac$, as required.

We're nearly done.  Put the three triangles together.

\begin{center} \includegraphics [scale=0.5] {pyth2c.png} \end{center}

The lower left and right corners are places where two vertices come together.  By complementary angles, these sum to be right angles.  

Since there are two other right angles at the vertices of the quadrilateral, it is a rectangle.  Therefore the place where three vertices come together is a straight line, which we can also verify because there are two complementary angles plus a right angle.

Since the composite figure is a rectangle, opposing sides are equal, so $a^2 + b^2 = c^2$.

$\square$

\subsection*{problem}

Draw the line connecting the vertices and draw diagonals to show that the resulting quadrilateral is a rectangle.

\subsection*{problem}

\begin{center} \includegraphics [scale=0.5] {similar23.png} \end{center}

The problem states that the bases are parallel ($XY \parallel BC$) and also that the subdivision produces \emph{equal areas}.  We are asked to find the ratio $AX:XB$.

\emph{Solution}.

Recall from a previous chapter that for two similar triangles, the altitudes to corresponding sides are in the same ratio as any of the three pairs of sides themselves.  

The altitudes $h$ and $H$ to $BC$ (not drawn) are also in the same ratio as the sides, so let $H = kh$ --- $H$ lies on $BC$.

Then twice the area of the top triangle is $AX \cdot h$ and twice the area of the whole is $AB \cdot H = k AB \cdot h$

We have that the whole is twice the smaller area so the ratio is equal to $2$:
\[ \frac{k AB \cdot h}{AX \cdot h} = 2 \]
But $AB/AX$ is also equal to $k$ so we have that $k^2 = 2$ and $k = \sqrt{2}$.

However, we are not asked for $k$.  Instead, we want the ratio of $AX$ to the smaller piece along the bottom. Let $AX = a$.  Then the whole is $A = ka$.  The difference is the small piece, $XB = A - a = a(k - 1)$.  We need
\[ \frac{a}{a(k - 1)} = \frac{1}{k - 1}  = \frac{1}{\sqrt{2} - 1} \]

\subsection*{problem}

The figure below is from Acheson's wonderful book aptly titled \emph{The Wonder Book of Geometry}.  He shows this problem, which he says "[goes] back to at least AD 850, when it appeared in a textbook by the Indian mathematician Mahavira."

\begin{center} \includegraphics [scale=0.5] {Acheson_ladders.png} \end{center}

Looking down an alleyway, you see two ladders arranged as shown and wonder about the point where they cross, at height $h$ and distances $c_1$ and $c_2$ from the edges of the alley, where the width of the alley is $c = c_1 + c_2$.

By similar triangles
\[ \frac{c_1}{h} = \frac{c}{b} \]

Can you see why?

Going the opposite direction
\[ \frac{c_2}{h} = \frac{c}{a} \]

Adding the two equations and substituting for $c_1 + c_2$:
\[ \frac{c}{h} = \frac{c}{a} + \frac{c}{b} \]

Thus
\[ \frac{1}{h} = \frac{1}{a} + \frac{1}{b} \]

That's a simple and interesting result.  $h$ depends only on $a$ and $b$ and not on $c, c_1$, or $c_2$.

If you think about it, you should see that in the previous problem we never used the information that the sides and the height were vertical, only parallel.

\begin{center} \includegraphics [scale=0.4] {similar25.png} \end{center}
Work through the same proof for the figure above to show that $1/x + 1/y = 1/z$.

\subsection*{problem}

Here's a problem from the web.  Given that $AC \parallel EF$ and $AB \parallel DF$.

We are to prove that the sum of the altitudes of the small triangles is equal to the altitude of the large one.

\begin{center} \includegraphics [scale=0.4] {prob_similar_tri2.png} \end{center}

Informal solution:  The smaller triangles are all similar to $\triangle ABC$ by the alternate interior angles and vertical angle theorems.

For similar triangles, not only are the sides in the same ratio to each other, but so are other measures like the altitudes to a particular side.  So if we label the bases $b_1$ etc., collectively $b_i$ , then we have that
\[ \frac{b_i}{h_i} = \frac{b}{h} \]
\[ b_i = b \cdot h_i/h \]
for each of the $b_i$.

But the sum of the $b_i$ is simply equal to $b$ so
\[ b_1 + b_2 + b_3 = b \cdot (h_1/h + h_2/h + h_3/h) \]
\[ b = b \cdot (h_1/h + h_2/h + h_3/h) \]
\[ 1 = h_1/h + h_2/h + h_3/h \]
\[ h = h_1 + h_2 + h_3 \]

\subsection*{problem}

Given two triangles that are not similar, where the ratio $AD/AB = r$ and $AE/AC = s$.  We are asked to show that 

\[ \frac{\triangle_{ADE}}{\triangle_{ABC}} = rs \] 

\begin{center} \includegraphics [scale=0.4] {similarity_by_area2.png} \end{center}

\emph{Solution}.

Draw $DF \parallel BC$.  Now $\triangle ADF \sim \triangle ABC$ so $AF/AC = r$.

By our previous result
\[ \frac{\triangle_{ADF}}{\triangle_{ABC}} = r^2 \] 

Since $AE/AC = s$ and $AF/AC = r$, $AE/AF = s/r$.  As triangles with a common vertex and bases in that proportion, these areas are in the same proportion:

\[ \frac{\triangle_{ADE}}{\triangle_{ADF}} = \frac{s}{r} \] 

Multiply the two ratios together to obtain

\[ \frac{\triangle_{ADE}}{\triangle_{ABC}} = sr \] 

$\square$

\emph{Solution}.  (alternate).

\begin{center} \includegraphics [scale=0.4] {similarity_by_area2.png} \end{center}

Looking ahead to trigonometry, in any triangle, twice the area can be computed as the product of two sides flanking an angle times the \emph{sine} of the angle.  (The reason is that the altitude to one side, divided by the length of the second side, is defined to be the sine of the angle.)

\[ 2 (ADE) = AD \cdot AE \cdot \sin \angle DAE \]
\[ 2 (ABC) = AB \cdot AC \cdot \sin \angle BAC \]

But $\angle DAE = \angle BAC$, so they have the same sine, and the ratio of areas is just
\[ \frac{AD \cdot AE}{AB \cdot AC} = rs \]

$\square$

To rework this proof in terms of familiar concepts, draw the altitude from $D$ to $AE$, and also the one from $B$ to $AC$

They form similar right triangles including the angle at $A$.  The altitudes scale like $AD/AB = r$.  But the bases scale like $AE/AC = s$.  

And area scales like the product of the two, namely, $rs$.

\subsection*{pyramid height}

As we said earlier, Thales was from Miletus and he lived around 600 BC.  Thales is believed to have traveled extensively and was likely of Phoenician heritage.  As you probably know, the Phoenicians were famous sailors who founded many settlements around the Mediterranean.  

They competed with the mainland Greeks and later with the Romans for colonies, and their major city, Carthage, was destroyed much later by the Romans, in the third Punic War.  Hannibal rode his famous elephants over the Alps in the second Punic war.

During his travels, Thales went to Egypt, home to the great pyramids at Giza, which were already ancient then.  They had been built about 2560 BC (dated by reference to Egyptian kings) and were already 2000 years old at that time!

The story is that Thales asked the Egyptian priests about the height of the Great Pyramid of Cheops, and they would not tell him.  So he set about measuring it himself.  He used similar triangles.  I'm sure he wrote down his answer, but I'm not aware that it survives.  The current height is 480 feet.

\begin{center} \includegraphics [scale=0.25] {Thales_theorem_6.png} \end{center}

\end{document}