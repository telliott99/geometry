\documentclass[11pt, oneside]{article} 
\usepackage{geometry}
\geometry{letterpaper} 
\usepackage{graphicx}
	
\usepackage{amssymb}
\usepackage{amsmath}
\usepackage{parskip}
\usepackage{color}
\usepackage{hyperref}

\graphicspath{{/Users/telliott/Github-math/figures/}}
% \begin{center} \includegraphics [scale=0.4] {gauss3.png} \end{center}

\title{Introduction}
\date{}

\begin{document}
\maketitle
\Large

%[my-super-duper-separator]

\begin{center} \includegraphics [scale=0.15] {isosceles9.png} \end{center}

The following statements about isosceles triangles are equivalent:
\begin{enumerate}
\item two sides equal
\item base angles equal
\item medians ($m$) equal
\item angle bisectors ($l$) equal
\item altitudes ($h$) equal
\item the bisector, median and altitude at $b$ coincide
\end{enumerate}

where in (1-5) the medians, bisectors and altitudes in question are those extending to the equal sides.

The figure shows (right panel) the case for the medians, but we will use the same labels for the other cases in the discussion below.

The question is then how to prove these one from the other, for example 1 $\Rightarrow$ 3 or 5 $\Rightarrow 2$.  Of course, we already have 1 $\Rightarrow$ 2 and 2 $\Rightarrow$ 1.  

In what we might call the forward direction \{ 1,2 \} $\Rightarrow$ \{ 3,4,5,6 \}, these proofs are quite easy.  I encourage you to try them before reading further.

The converse proofs are another matter.

\subsection*{forward}

\label{sec:more_isosceles_fwd}

\begin{center} \includegraphics [scale=0.16] {isosceles10.png} \end{center}

1 $\Rightarrow$ 3

\emph{Proof}. $AF$ and $CH$ are medians in $\triangle ABC$.  Given $AB = CB$ and $\angle A = \angle C$, it follows that $AH = CF$ (by the definition of median) and thus by SAS we have $\triangle ACH \cong \triangle CAF$.  Hence $AF = CH$.  $\square$

1 $\Rightarrow$ 4

\emph{Proof}. $AF$ and $CH$ are angle bisectors in $\triangle ABC$.  Given $\angle A = \angle C$, the half-angles are also equal, so by ASA we have $\triangle ACH \cong \triangle CAF$.  Hence $AF = CH$.  $\square$

1 $\Rightarrow$ 5

\emph{Proof}.  $AF$ and $CH$ are altitudes in $\triangle ABC$.  Twice the area of $\triangle ABC$ may be computed as $AF \cdot CB = CH \cdot AB$.  But $AB = CB$.   Hence $AF = CH$.  $\square$

\subsection*{converse}

\label{sec:more_isosceles_con}

One of the converse proofs is also easy.

5 $\Rightarrow$ 1

\emph{Proof}.  $AF$ and $CH$ are altitudes in $\triangle ABC$.  Twice the area of $\triangle ABC$ may be computed as $AF \cdot CB = CH \cdot AB$.  But $AF = CH$.   Hence $AB = CB$.  $\square$

\begin{center} \includegraphics [scale=0.16] {isosceles10.png} \end{center}

4 $\Rightarrow$ 1

If the angle bisectors are equal, then the triangle is isosceles.  This is a famous theorem, and there is a whole chapter about it here:  \hyperref[sec:Steiner_Lehmus_Theorem]{\textbf{Steiner-Lehmus theorem}}.

3 $\Rightarrow$ 1

For the last one, we look ahead to the $\hyperref[sec:law_of_cosines]{\textbf{Law of cosines}}$.

\emph{Proof}.

We have that the medians are equal, namely, $AF = CH$.  Let them be equal to $m$.  Using the law of cosines, compute the length $m$ squared of the side opposite $\angle B$ in two different triangles.

Let side $a = BC$ (the side opposite $\angle A$, and similarly side $c = AB$ (the side opposite $\angle C$).

So $BF = a/2$ and $BH = c/2$.  Then
\[ m^2 = a^2 + (c/2)^2 - 2a \cdot c/2 \cdot \cos B \]
\[ m^2 = c^2 + (a/2)^2 - 2c \cdot a/2 \cdot \cos B \]

The last terms on the right-hand side are equal, namely
\[ ac \cdot \cos B = ac \cdot \cos B \]

Setting the two expressions equal, it follows that
\[ a^2 + (c/2)^2 =  c^2 + (a/2)^2 \]
\[ \frac{3}{4} a^2 = \frac{3}{4} c^2 \]
\[ a^2 = c^2 \]

Since $a$ and $c$ are both lengths, we may take the positive square roots for each, and thus $a = c$. 

$\square$

\subsection*{median, bisector and altitude at $b$}

\begin{center} \includegraphics [scale=0.16] {isosceles11.png} \end{center}

Let $b$ be the side opposite $\angle B$ as usual.

\textbf{1 $\Rightarrow$ 6}

We have that $AB = CB$ and $\angle A = \angle C$.

\emph{Proof}.  Let $BG$ be the median to $AC$ such that $AG = GC$  Then $\triangle ABG \cong \triangle CBG$ by SSS.  Thus $\angle B$ is bisected and $\angle AGB = \angle CGB$ and both are right angles.  $\square$

\emph{Proof}.  Let $BG$ be the altitude to $AC$ such that $\angle AGB = \angle CGB$ and both are right angles.  Then $\triangle ABG \cong \triangle CBG$ by HL.  Thus $\angle B$ is bisected and $AG = GC$.   $\square$

\emph{Proof}.  Let $BG$ bisect $\angle B$.  Then $\triangle ABG \cong \triangle CBG$ by SAS.  Thus $\angle AGB = \angle CGB$ and both are right angles, and $AG = GC$.   $\square$

\textbf{6 $\Rightarrow$ 1}

We will prove that if any two of these statements are both true, then $a = c$.   In what follows, if we have  $\triangle ABG \cong \triangle CBG$, then (1) and (2) follow immediately.

\emph{Proof}.  If the bisector and the altitude at $b$ coincide, then $\triangle ABG \cong \triangle CBG$ by ASA. $\square$

\emph{Proof}.  If the median and the altitude at $b$ coincide: then $\triangle ABG \cong \triangle CBG$ by SAS.$\square$

Finally, suppose the bisector and the median at $b$ coincide.  Draw perpendiculars from the midpoint of $b$ to each of sides $a$ and $c$.

\begin{center} \includegraphics [scale=0.16] {isosceles12.png} \end{center}

\emph{Proof}.

$BG$ bisects $\angle B$, $\angle GDB = \angle GEB$ and both are right angles, therefore $\angle BGE = \angle BGD$ by sum of angles.  The hypotenuse $BG$ is shared.  It follows that $\triangle BEG \cong \triangle BDG$ by ASA.  Thus $GE = GD$.

Then (since $AG = GC$), $\triangle AEG \cong \triangle CDG$ by HL.  It follows that $\angle A = \angle C$ and then by 2 $\Rightarrow$ 1, we have $a = c$.

$\square$

I found much of this material in Byer (see $\hyperref[sec:list_of_references]{\textbf{References}}$).

\end{document}