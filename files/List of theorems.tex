\documentclass[11pt, oneside]{article} 
\usepackage{geometry}
\geometry{letterpaper} 
\usepackage{graphicx}
	
\usepackage{amssymb}
\usepackage{amsmath}
\usepackage{parskip}
\usepackage{color}
\usepackage{hyperref}

\graphicspath{{/Users/telliott/Dropbox/Github-math/figures/}}
% \begin{center} \includegraphics [scale=0.4] {gauss3.png} \end{center}

\title{List of theorems}
\date{}

\begin{document}
\maketitle
\Large

%[my-super-duper-separator]

\subsection*{proofs of the Pythagorean theorem}

$\circ$ \ \ \hyperref[sec:Pythagoras_similar_triangles]{\textbf{Pythagorean Thm:  similar triangles}}

$\circ$ \ \ \hyperref[sec:Euclid47]{\textbf{Pythagorean Thm: Euclid I.47 }}

$\circ$ \ \ \hyperref[sec:Pythagoras_scaled_triangles]{\textbf{Pythagorean Thm:  scaled triangles}}

$\circ$ \ \ \hyperref[sec:pthm_sum_angles]{\textbf{Pythagorean Thm:  sum of angles}}

$\circ$ \ \ \hyperref[sec:Pythagoras_area]{\textbf{Pythagorean Thm:  area}}

$\circ$ \ \ \hyperref[sec:Garfield]{\textbf{Pythagorean Thm: Garfield }}

$\circ$ \ \ \hyperref[sec:PProof_Pappus]{\textbf{Pythagorean Thm:  Pappus}}

$\circ$ \ \ \hyperref[sec:PProof_chords]{\textbf{Pythagorean Thm:  crossed chords}}

$\circ$ \ \ \hyperref[sec:star_of_david]{\textbf{Pythagorean Thm:  Star of David, Anderson}}

$\circ$ \ \ \hyperref[sec:PProof_Ptolemy]{\textbf{Pythagorean Thm: Ptolemy }}

$\circ$ \ \ \hyperref[sec:Condit]{\textbf{Pythagorean Thm:  Condit}}

$\circ$ \ \ \hyperref[sec:Pthm_Tuan]{\textbf{Pythagorean Thm:  Tuan}} (fancy proof)

$\circ$ \ \ \hyperref[sec:Quorra]{\textbf{Pythagorean Thm:  Quorra corollary}}

$\circ$ \ \ \hyperref[sec:Pythagorean_theorem_converse]{\textbf{Pythagorean Thm:  converse}}


\subsection*{proofs from Euclid}

$\circ$ \ \ \hyperref[sec:Euclid1]{\textbf{construct an equilateral triangle}} (Euclid I.1)

$\circ$ \ \ \hyperref[sec:Euclid4]{\textbf{side angle side (SAS)}} (Euclid I.4)

$\circ$ \ \ \hyperref[sec:Euclid5]{\textbf{isosceles triangle theorem}} (Euclid I.5:  equal sides $\rightarrow$ angles)

$\circ$ \ \ \hyperref[sec:Euclid6]{\textbf{isosceles triangle theorem converse}} (Euclid I.6:  equal angles $\rightarrow$ sides)

$\circ$ \ \ \hyperref[sec:Euclid9]{\textbf{angle bisection}} (Euclid I.9)

$\circ$ \ \ \hyperref[sec:Euclid10]{\textbf{perpendicular bisector}} (Euclid I.10)

$\circ$ \ \ \hyperref[sec:Euclid11]{\textbf{perpendicular through a point}} (Euclid I.11)

$\circ$ \ \ \hyperref[sec:Euclid12]{\textbf{perpendicular to a point}} (Euclid I.12)

$\circ$ \ \ \hyperref[sec:Euclid16]{\textbf{external angle inequality}} (Euclid I.16)

$\circ$ \ \ \hyperref[sec:Euclid18]{\textbf{longer side $\rightarrow$ larger angle}}  (Euclid I.18)

$\circ$ \ \ \hyperref[sec:Euclid19]{\textbf{larger angle $\rightarrow$ longer side}} (Euclid I.19)

$\circ$ \ \ \hyperref[sec:triangle_inequality]{\textbf{triangle inequality}} (Euclid I.20)

$\circ$ \ \ \hyperref[sec:hinge_theorem]{\textbf{hinge theorem}} (Euclid I.24)

$\circ$ \ \ \hyperref[sec:Euclid31]{\textbf{line parallel to another line}} (Euclid I.31)

$\circ$ \ \ \hyperref[sec:Euclid_I_35]{\textbf{parallelogram area}} (Euclid I.35)

$\circ$ \ \ \hyperref[sec:find_circle_center]{\textbf{find circle center}} (Euclid III.1)

$\circ$ \ \ \hyperref[sec:inscribed_angle_theorem]{\textbf{inscribed angle theorem}} (Euclid III.20)

$\circ$ \ \ \hyperref[sec:angles_on_same_arc]{\textbf{same arc $\rightarrow$ equal angles}} (Euclid III.21)

$\circ$ \ \ \hyperref[sec:quadrilateral_supplementary]{\textbf{quadrilateral supplementary theorem}} (Euclid III.22)

$\circ$ \ \ \hyperref[sec:Thales_theorem]{\textbf{Thales' theorem}} (Euclid III.31)

$\circ$ \ \ \hyperref[sec:tangent_chord_theorem]{\textbf{tangent-chord theorem}} (Euclid III.32)

$\circ$ \ \ \hyperref[sec:Euclid6_2]{\textbf{similarity:  AAA $\rightarrow$ equal ratios}} (Euclid VI.2)

$\circ$ \ \ \hyperref[sec:Euclid6_9]{\textbf{equal divisions of a line segment}} (Euclid VI.9)


\subsection*{proofs of other theorems}

$\circ$ \ \  \hyperref[sec:alternate_interior_angle_theorem]{\textbf{alternate interior angles}}

$\circ$ \ \ \hyperref[sec:angle_bisector_theorem]{\textbf{angle bisector theorem}} (right triangle)

$\circ$ \ \ \hyperref[sec:generalized_angle_bisector_theorem]{\textbf{angle bisector theorem}} (general)

$\circ$ \ \ \hyperref[sec:ASA]{\textbf{ASA for congruence}}

$\circ$ \ \ \hyperref[sec:ASA_contradiction]{\textbf{angle side angle (ASA)}} (by contradiction)

$\circ$ \ \ \hyperref[sec:Apollonius_theorem]{\textbf{Apollonius theorem}}

$\circ$ \ \ \hyperref[sec:circle_area_Archimedes]{\textbf{area of a circle}} (Archimedes proof)

$\circ$ \ \ \hyperref[sec:area_ratio_theorem]{\textbf{area ratio theorem}}

$\circ$ \ \ \hyperref[sec:bisector_equidistant_sides]{\textbf{bisector equidistant from sides}}

$\circ$ \ \ \hyperref[sec:bisector_equidistant_sides_converse]{\textbf{equidistant from sides $\rightarrow$ bisector}}

$\circ$ \ \ \hyperref[sec:bisected_hypotenuse]{\textbf{bisected hypotenuse}} (right triangle, by contradiction)

$\circ$ \ \ \hyperref[sec:Ceva_theorem]{\textbf{Ceva's theorem}}

$\circ$ \ \ \hyperref[sec:chord_segments]{\textbf{crossed chords}} (product of lengths)

$\circ$ \ \ \hyperref[sec:complementary_angle_theorem]{\textbf{complementary angles}}

$\circ$ \ \ \hyperref[sec:diameter_of_a_circle]{\textbf{diameter divides circle in half}}

$\circ$ \ \ \hyperref[sec:diameters_form_rectangle]{\textbf{diameters form a rectangle}}

$\circ$ \ \ \hyperref[sec:equal_arcs_equal_chords]{\textbf{equal arcs $\iff$ equal chords}}

$\circ$ \ \ \hyperref[sec:eyeball_theorem]{\textbf{eyeball theorem}}

$\circ$ \ \ \hyperref[sec:excircle_theorems]{\textbf{excircle theorems}}

$\circ$ \ \ \hyperref[sec:exterior_angle_sum_theorem]{\textbf{exterior angle sum theorem}}

$\circ$ \ \ \hyperref[sec:external_angle_theorem]{\textbf{external angle theorem}}

$\circ$ \ \ \hyperref[sec:extraordinary_property]{\textbf{extraordinary property of the circle}}

$\circ$ \ \ \hyperref[sec:Gauss_orthocenter]{\textbf{orthocenter exists}} (Gauss)

$\circ$ \ \ \hyperref[sec:Heron_formula]{\textbf{Heron's formula}}

$\circ$ \ \ \hyperref[sec:Heron_formula_excircles]{\textbf{Heron's formula by excircles}}

$\circ$ \ \ \hyperref[sec:Heron_formula_Heron]{\textbf{Heron's formula, Heron's proof}}

$\circ$ \ \ \hyperref[sec:SSA_in_right]{\textbf{hypotenuse-leg in a right triangle (HL)}}

$\circ$ \ \ \hyperref[sec:hypotenuse_longest]{\textbf{hypotenuse longest side in a triangle}}

$\circ$ \ \ \hyperref[sec:incenter]{\textbf{incenter}} (incenter:  angle bisectors meet at a point)

$\circ$ \ \ \hyperref[sec:equal_angles_same_arc]{\textbf{inscribed angle theorem}}

$\circ$ \ \ \hyperref[sec:peripheral_angle]{\textbf{inscribed angle theorem}} (on a circle is one-half central angle)

$\circ$ \ \  \hyperref[sec:isosceles_triangle_theorem]{\textbf{isosceles triangle theorem}} (sides $\rightarrow$ angles)

$\circ$ \ \  \hyperref[sec:isosceles_converse]{\textbf{isosceles triangle theorem}} (angles $\rightarrow$ sides)

$\circ$ \ \ \hyperref[sec:law_of_cosines]{\textbf{Law of cosines}}

$\circ$ \ \ \hyperref[sec:law_of_cosines_algebraic]{\textbf{Law of cosines, algebraic proof}}

$\circ$ \ \ \hyperref[sec:LOC_by_Ptolemy]{\textbf{Law of cosines, Ptolemy}}

$\circ$ \ \ \hyperref[sec:Menelaus_theorem]{\textbf{Menelaus' theorem}}

$\circ$ \ \ \hyperref[sec:right_triangle_midpoint_theorem]{\textbf{midpoint theorem}} (right triangle)

$\circ$ \ \ \hyperref[sec:de_Moivre_theorem]{\textbf{de Moivre's theorem}}

$\circ$ \ \ \hyperref[sec:Newton_altitude]{\textbf{orthocenter exists}}  (Newton)

$\circ$ \ \ \hyperref[sec:nine_point_circle]{\textbf{Nine point circle}}

$\circ$ \ \ \hyperref[sec:PProof_Pappus]{\textbf{Pappus parallelogram theorem}}

$\circ$ \ \ \hyperref[sec:parallelogram_theorems]{\textbf{parallelogram theorems}}

$\circ$ \ \ \hyperref[sec:one_pair_of_sides]{\textbf{special parallelogram theorem}} (one pair of sides)

$\circ$ \ \ \hyperref[sec:perpendicular_bisector_of_a_chord]{\textbf{circumcenter}} (perpendicular bisectors of a chord is diameter)

$\circ$ \ \ \hyperref[sec:circumcenter]{\textbf{circumcenter}} (perpendicular bisectors meet at a point)

$\circ$ \ \ \hyperref[sec:Pi_is_a_constant]{\textbf{Pi is a constant}}

$\circ$ \ \ \hyperref[sec:pizza_proof]{\textbf{area of a circle}} (Pizza proof)

$\circ$ \ \ \hyperref[sec:Ptolemy]{\textbf{Ptolemy's theorem, by cutting}}

$\circ$ \ \ \hyperref[sec:Ptolemy_similar_triangles]{\textbf{Ptolemy's theorem, similar triangles}}

$\circ$ \ \ \hyperref[sec:Ptolemy_switch_sides]{\textbf{Ptolemy's theorem, switch sides}}

$\circ$ \ \ \hyperref[sec:Ptolemy_inversion]{\textbf{Ptolemy's theorem, by inversion}}

$\circ$ \ \ \hyperref[sec:rectangle_side_on_a_circle]{\textbf{rectangle in a circle}}

$\circ$ \ \ \hyperref[sec:right_angle_largest]{\textbf{right angle is largest in a triangle}}

$\circ$ \ \ \hyperref[sec:secant_tangent_theorem]{\textbf{secant tangent theorem}}

$\circ$ \ \ \hyperref[sec:shortest_distance_to_line]{\textbf{shortest distance from a point to a line}}

$\circ$ \ \ \hyperref[sec:equal_supplementary_angles]{\textbf{equal supplementary angles}}

$\circ$ \ \ \hyperref[sec:quadrilateral_supplementary]{\textbf{cyclic quadrilateral}} (opposing angles are supplementary)

$\circ$ \ \ \hyperref[sec:similar_right_triangles]{\textbf{similar right triangles}}

$\circ$ \ \ \hyperref[sec:midpoint_theorem]{\textbf{midpoint theorem}} (similar triangles)

$\circ$ \ \  \hyperref[sec:similarity_equal_ratios]{\textbf{similar triangles}} (ratio of sides)

$\circ$ \ \  \hyperref[sec:similarity_right_to_all_triangles]{\textbf{similar triangles}} (right triangle composition)

$\circ$ \ \ \hyperref[sec:similarity_theorem]{\textbf{AAA similarity theorem}} (Kiselev)

$\circ$ \ \ \hyperref[sec:SAS]{\textbf{SAS for congruence}}

$\circ$ \ \ \hyperref[sec:hinge_theorem]{\textbf{SAS inequality, hinge theorem}}

$\circ$ \ \ \hyperref[sec:SAS_similar]{\textbf{SAS to establish similarity}}

$\circ$ \ \ \hyperref[sec:sqrt_two]{\textbf{square root of 2 is irrational}}

$\circ$ \ \ \hyperref[sec:SSS_implies_SAS]{\textbf{SSS implies SAS}}

$\circ$ \ \ \hyperref[sec:Steiner_Lehmus_Theorem]{\textbf{Steiner-Lehmus theorem}}

$\circ$ \ \ \hyperref[sec:Stewarts_theorem]{\textbf{Stewart's theorem}}

$\circ$ \ \ \hyperref[sec:supplementary_angle_theorem]{\textbf{supplementary angle theorem}}

$\circ$ \ \ \hyperref[sec:two_supplementary_equal_two_right]{\textbf{supplementary angles equal to two right angles}}

$\circ$ \ \  \hyperref[sec:triangle_sum_theorem]{\textbf{sum of angles}}

$\circ$ \ \ \hyperref[sec:tangent_one_point]{\textbf{tangent theorem}} (right angle $\rightarrow$ touches one point)

$\circ$ \ \ \hyperref[sec:tangent_chord_theorem]{\textbf{tangent-chord theorem}}

$\circ$ \ \ \hyperref[sec:tangent_perpendicular]{\textbf{tangent theorem}} (touches one point $\rightarrow$ right angle)

$\circ$ \ \ \hyperref[sec:tangent_construction]{\textbf{tangent construction}}

$\circ$ \ \ \hyperref[sec:Thales_theorem]{\textbf{Thales' circle theorem}} (right angle in a semi-circle)

$\circ$ \ \ \hyperref[sec:Thales_circle_theorem_converse]{\textbf{Thales circle theorem:  converse}}

$\circ$ \ \ \hyperref[sec:triangle_area]{\textbf{triangular area}}

$\circ$ \ \ \hyperref[sec:triangle_inequality]{\textbf{triangle inequality}} (triangle inequality)

$\circ$ \ \ \hyperref[sec:two_angles_similar]{\textbf{triangles are similar if two angles equal}}

$\circ$ \ \ \hyperref[sec:Varignon_theorem]{\textbf{Varignon's theorem}}

$\circ$ \ \ \hyperref[sec:vertical_angle_theorem]{\textbf{vertical angle theorem}}


\end{document}