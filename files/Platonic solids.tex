\documentclass[11pt, oneside]{article} 
\usepackage{geometry}
\geometry{letterpaper} 
\usepackage{graphicx}
	
\usepackage{amssymb}
\usepackage{amsmath}
\usepackage{parskip}
\usepackage{color}
\usepackage{hyperref}

\graphicspath{{/Users/telliott/Dropbox/Github-Math/figures/}}
% \begin{center} \includegraphics [scale=0.4] {gauss3.png} \end{center}

\title{Area}
\date{}

\begin{document}
\maketitle
\Large

%[my-super-duper-separator]

This last part doesn't exactly fit in the book, because it's solid geometry and we don't have much of that, but it's one of the most beautiful proofs in Greek geometry and I would like to include it.  So here goes.

\url{https://en.wikipedia.org/wiki/Platonic_solid}

\begin{quote}
In three-dimensional space, a Platonic solid is a regular, convex polyhedron. It is constructed by congruent (identical in shape and size) regular (all angles equal and all sides equal) polygonal faces with the same number of faces meeting at each vertex. Five solids meet these criteria.
\end{quote}

\begin{center} \includegraphics [scale=0.5] {platonic_solids.png} \end{center}
These are:  (i) tetrahedron, (ii) cube, (iii) octagon, (iv) dodecagon, and (v) icosahedron.

There is a wonderful, simple proof that there are only five of them.  Any solid requires at least three sides meeting at each vertex, otherwise the joint between two sides can just flap, like a hinge.  Furthermore, the total of all the vertex angles added up must be less than $360$ degrees, since otherwise the figure would be planar, not 3-dimensional.

$\circ$  Three equilateral triangles total $60 \times 3 = 180$, four total $60 \times 4 = 240$ and five total $60 \times 5 = 300$.  Six would be a hexagon lying in the plane.  

$\circ$  Three squares total $90 \times 3 = 270$, while four give a square array in the plane.  

$\circ$  Finally, three pentagons give $108 \times 3 = 324$.  And that's it.  Three hexagons would give $120 \times 3 = 360$, which gives an array in the plane.

Proving that all the angles and side lengths come out correctly, so that the possible solids actually can be constructed is another matter, however.  Euclid devotes book XIII of \emph{The Elements} to this:

\url{https://mathcs.clarku.edu/~djoyce/elements/bookXIII/bookXIII.html#props}


\end{document}