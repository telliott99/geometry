\documentclass[11pt, oneside]{article} 
\usepackage{geometry}
\geometry{letterpaper} 
\usepackage{graphicx}
	
\usepackage{amssymb}
\usepackage{amsmath}
\usepackage{parskip}
\usepackage{color}
\usepackage{hyperref}

\graphicspath{{/Users/telliott/Dropbox/Github-math/figures/}}
% \begin{center} \includegraphics [scale=0.4] {gauss3.png} \end{center}

\title{Beginnings of trigonometry}
\date{}

\begin{document}
\maketitle
\Large

%[my-super-duper-separator]

\subsection*{double angle formulas}

Draw a unit circle with central angle $\theta$ and then double the central angle below the horizontal diameter.  Let $x$ be the base of the two smaller triangles, the side adjacent to angle $\theta$, and $y$ the side opposite.

\begin{center} \includegraphics [scale=0.4] {trig_beg_3.png} \end{center}.

Then by the basic definitions of trigonometry, we have that $x = \cos \theta$ and $y = \sin \theta$.  The area of each small triangle is $xy/2$ and the area of the two combined is simply
\[ A = xy = \sin \theta \cos \theta \]

Now, rotate the triangle counter-clockwise until the red side on the right lies along the diagonal.  The dotted vertical line is clearly $\sin 2 \theta$ and since the base is length $1$, the area of that same triangle is now

\[ A = \frac{1}{2} \cdot \sin 2 \theta \]
Combining the results
\[ \sin 2 \theta = 2 \sin \theta \cos \theta \]

This is the first double angle formula, for sine.  We claim that it still works even if the sum is greater than a right angle but keep things simple by not talking about it more.

\subsection*{double angle for cosine}
Consider this right triangle with an inscribed angle of $\phi$.  Label the sides of a triangle containing the central angle $2 \phi$ as $x$ and $y$ to simplify the figure.  Let $r = 1$ so then
\[ y = \sin 2 \phi, \ \ \ \ \ \ x = \cos 2 \phi \]

Whenever we drop an altitude in a right triangle, the smaller triangles are similar to the original.  This accounts for the second $\angle \phi$ in the figure.
\begin{center} \includegraphics [scale=0.4] {trig_beg_6.png} \end{center}
We note for later that the base of the altitude divides the base into $r + x$ and $r - x$.

Using the small triangle with angle $\phi$
\[ \frac{y}{2 \sin \phi} = \cos \phi \]
\[ y = 2 \sin \phi \cos \phi \]
Since $y = \sin 2 \phi$, this is just what we had before.

Along the base of the triangle extending from the left, we have (again, $r = 1$):
\[ \frac{r + x}{2 \cos \phi} = \frac{1 + x}{2 \cos \phi} = \cos \phi \]
\[ \frac{1 + x}{2} = \cos^2 \phi \]
But (from the small triangle again)
\begin{center} \includegraphics [scale=0.4] {trig_beg_6.png} \end{center}

\[ \frac{r - x}{2 \sin \phi} = \frac{1 - x}{2 \sin \phi} = \sin \phi \]
\[ \frac{1 - x}{2} = \sin^2 \phi \]
And then
\[  \frac{1 + x}{2} - \frac{1 - x}{2} = x  \]
but $x = \cos 2 \phi$ so
\[ \cos 2 \phi = \cos^2 \phi - \sin^2 \phi \]
Note also that
\[ \cos^2 \phi + \sin^2 \phi = \frac{1 + x}{2} + \frac{1 - x}{2} = 1 \]
which is true for any angle $\phi$, a restatement of the Pythagorean theorem.

Here is the same proof but scaled so that the diameter has unit length and we use sine and cosine labels rather than $x$ and $y$.  The details are left to you.
\begin{center} \includegraphics [scale=0.4] {half_angle_3.png} \end{center}

\subsection*{half-angle formulas}

The half-angle formulas are easy to derive from the double angle cosine formula, because the sine and cosine terms are separated:

\[ \cos 2A = \cos^2 A - \sin^2 A \]
Just use our favorite identity ($\sin^2 A + \cos^2 A = 1$) to obtain
\[ \cos 2A = \cos^2 A - (1 - \cos^2 A) \]
or
\[ \cos 2A = (1 - \sin^2 A) - \sin^2 A \]

We'll do the sine first
\[ 2\sin^2 A = 1 - \cos 2A \]
\[ \sin A = \sqrt{\frac{1 - \cos 2A}{2}} \]
which is easily verified using the values for $30^{\circ}$ and $60^{\circ}$.

On the other hand:
\[ \cos 2A = \cos^2 A + \cos^2 A - 1 \]
Solve for $\cos A$:
\[ \cos A = \sqrt{\frac{1 + \cos 2A}{2}} \]

This is called the half-angle formula since $A$ is one-half of $2A$.  That the sum of the squares is equal to $1$ is also easily verified.  We see that the square roots go away, the $\cos 2A$ terms will cancel and we end up with $1/2 + 1/2$.

\subsection*{some algebra}

You might be tempted to try to derive a formula for $\sin 2A$ from that for $\cos 2A$ (or vice-versa), using our favorite identity.  Let us re-write the formulas substituting $s$ for $\sin A$ and $c$ for $\cos A$:
\[ \sin 2A = 2sc, \ \ \ \ \ \  \cos 2A = c^2 - s^2 \]

They have different forms:  the first mixes $s$ and $c$, while the second has a difference of squares.  If we were to take
\[ (c + s)^2 = c^2 + 2sc + s^2 \]
\[ (c - s)^2 = c^2 - 2sc + s^2 \]
\[ (c + s)(c - s) = c^2 - s^2 \]

There is some of what we want, but in the formula for $\cos 2A$ one term is positive and one negative.  What's going on?

First, just verify what $\sin^2 2A + \cos^2 2A$ is equal to:
\[ = 4s^2c^2 + c^4 - 2s^2c^2 + s^4 \]
\[ = c^4 + 2s^2c^2 + s^4 \]
\[ = (s^2 + c^2)^2 = 1^2 = 1 \]
That checks out, and it contains a hint to the answer.  We used $4s^2c^2 - 2s^2c^2$ to convert $-2s^2c^2$ to the positive $2s^2c^2$.

Remember, to do the conversion we square the sine or cosine, subtract it from $1$ and then take the square root.  \[ \sin \theta = \sqrt{1 - \cos^2 \theta} \]
\[ \cos \theta = \sqrt{1 - \sin^2 \theta} \]

It is easier to see how this works using the first formula.  So we start with the formula for cosine of $2A$ and plug in:
\[ 1 - (c^2 - s^2)^2 = 1 - (c^4 - 2s^2c^2 + s^4) \]

Add (and subtract) $2s^2c^2$ in the parentheses!
\[ = 1 - (c^4 + 2s^2c^2 + s^4 - 4s^2c^2) \]
\[ = 1 - \ [ \ (c^2 + s^2)^2 - 4s^2c^2 \ ]  \]
\[ = 1 - 1 + 4s^2c^2 = 4s^2c^2 \]
and now the square root is easy and gives exactly what we need.

Then it becomes clear that we can add $(s^2 + c^2) - 1$ or $(s^2+c^2)^2 - 1$ to expressions and if they will simplify, it helps us.  Hence to obtain the formula for $\cos 2A$, starting with $1$ minus sine squared:
\[ 1 - 4s^2c^2 \]
\[ = 1 - 4s^2c^2 + (s^2 + c^2)^2 - 1 \]
\[ = - 4s^2c^2 + \ [ \ s^4 + 2s^2 c^2 - c^4 \ ]   \]
\[ = s^4 - 2s^2c^2 + c^4) \]
\[ = (s^2 - c^2)^2 = (c^2 - s^2)^2 \]
remembering that $(x - y)^2 = (y - x)^2$.

Take the square root to finish.

\subsection*{examples}
Restating the double-angle formula for sine
\[ \sin 2A = 2 \sin A \cos A \]

Plugging in
\[ \sin 60^{\circ} = 2 \cdot \frac{1}{2} \cdot \frac{\sqrt{3}}{2} = \frac{\sqrt{3}}{2}  \]
and
\[ \cos 2A = \cos^2 A - \sin^2 A \]
plugging in again
\[ \cos 60^{\circ} = ( \frac{\sqrt{3}}{2})^2 - (\frac{1}{2})^2 = \frac{3}{4} - \frac{1}{4} = \frac{1}{2} \]
They both look good.

We can also use the formulas to check supplementary angles (recall that $\sin 180 = 0$ and $\cos 180 = -1$):

\[ \sin 180 - A = \sin 180 \cos A - \sin A \cos 180 \]
\[ = 0 - (- \sin A) = \sin A \]
\[ \cos 180 - A = \cos 180 \cos A + \sin 180 \sin A \]
\[ = - \cos A + 0 = - \cos A \]

Of course, we hardly need algebra for the result about the sine of the supplementary angle.  Two inscribed angles with the supplementary arcs clearly have the same sine:
\begin{center} \includegraphics [scale=0.4] {supp.png} \end{center}
The total arc of the circle is $2 \pi$, while the two angles $\theta$ and $\phi$ together take up that entire arc.  As inscribed angles they sum to one-half or $\pi$.  Hence $\theta$ and $\phi$ are supplementary.

But we previously had
\[ L = d \sin \phi = 2r \sin \phi \]
where $L$ is the length of the dotted line, and if the circle is scaled appropriately, it \emph{is} $\sin \phi$.  But it is also $\sin \theta$.

\subsection*{equilateral triangle}

\begin{center} \includegraphics [scale=0.4] {trig_60.png} \end{center}
Here is another way of looking at $30^{\circ}$ and $60^{\circ}$.  Let the central angle $\theta$ be such that the triangle formed is equilateral.  

Now the central angle is equal to both of the base angles, and the measure of all these angles is $60^{\circ}$ or $\pi/3$.

Since $\triangle OQR$ is equilateral, the chord $QR$ is the same length as the radius.  It follows that the sine of the inscribed angle $\phi$ is $r/2r = 1/2$.  And of course the length of $PR$, which is 2 times the sine of $60^{\circ}$, is $\sqrt{3}$.

\subsection*{Archimedes and pi}

Later we will explore the math relating to Archimedes' method to approximate $\pi$, resulting in the bounds $3\ 10/71 < \pi < 3\ 1/7$.

The lower bound is about $3.1408$ and the upper bound about $3.1429$ while the true value is about $3.1416$.  (There is an interesting story relating to a fraction which is a much, much better estimate of $\pi$, namely $355/113$, good to six digits).

As you probably know, the method involves calculating the length of the perimeter of circumscribed and inscribed regular polygons. 

We can build on two basic results from the geometry of right triangles:
\begin{center} \includegraphics [scale=0.3] {pi12.png} \end{center}

The first is that the sum of the cotangent and cosecant for the original angle are equal to the cotangent of the half-angle:
\[ \frac{a}{f} + \frac{c}{f} = \frac{a}{d} \]
\[ \cot 2 \theta + \csc 2 \theta = \cot \theta \]

This follows from the angle bisector theorem, $a/d = c/e$, with a bit of manipulation.
\[ \frac{a}{c} = \frac{d}{e} \]
\[ \frac{a}{c} + \frac{c}{c} = \frac{d}{e} + \frac{e}{e} \]
\[ \frac{a + c}{d + e} = \frac{c}{e} = \frac{a}{d} \]
Adding components in the denominator of the first term:
\[ \frac{a + c}{f} = \frac{a}{d} \]

Another way to get there is to use the double angle formulas.  We'll abbreviate $\sin, \cos, \tan$ as $S,C,T$ and identify the values for the half-angle as lowercase.  The formulas are:
\[ S = 2sc  \]
\[ C = c^2 - s^2 = 2c^2 - 1 \] 

The cotangent of $2 \theta$ is then (cosine/sine):
\[ \frac{1}{T} = \frac{C}{S} = \frac{2c^2 - 1}{2sc} \]
\[ = \frac{1}{t} - \frac{1}{S} \]
which rearranges to give the result
\[ \frac{1}{t} = \frac{1}{T} + \frac{1}{S} \]

\end{document}