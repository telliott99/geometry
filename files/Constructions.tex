\documentclass[11pt, oneside]{article} 
\usepackage{geometry}
\geometry{letterpaper} 
\usepackage{graphicx}
	
\usepackage{amssymb}
\usepackage{amsmath}
\usepackage{parskip}
\usepackage{color}
\usepackage{hyperref}

\graphicspath{{/Users/telliott/Dropbox/Github-Math/figures/}}
% \begin{center} \includegraphics [scale=0.4] {gauss3.png} \end{center}

\title{Constructions}
\date{}

\begin{document}
\maketitle
\Large

%[my-super-duper-separator]

A number of the propositions in book 1 of Euclid concern constructions.

\begin{center} 
\includegraphics [scale=0.2] {straightedge.png} 
\includegraphics [scale=0.3] {compass.png} 
\end{center}

The tools we have are a straight-edge and a compass.  The compass is collapsible, meaning that it cannot be used to transfer distances since it loses its setting when lifted from the page.  This is a limitation Euclid solves in the second and third propositions of book I.  It's also important that the straight-edge is not a ruler, there is no way to measure distance by reference to marks on it.

Euclid was smart enough to know about compasses and how to set them.  The idea he had was this:  to make the fewest possible assumptions.  A non-collapsible compass was a luxury he didn't need, since he could accomplish the same end without it.

\subsection*{Euclid I.1:  equilateral triangle}

\label{sec:Euclid_I_1}

To construct an equilateral triangle on a given line segment.
\begin{center} \includegraphics [scale=0.4] {PI_1a.png} \end{center}

The first step is to draw two circles on centers $A$ and $B$.
\begin{center} \includegraphics [scale=0.3] {PI_1b.png} \end{center}

The circles are drawn with each radius equal to the line segment $AB$.  It is a property of circles that all points on the circle are at the same distance from the center.  Thus all points on the left-hand circle are equidistant from $A$, and all points on the second one are equidistant from $B$.  

Therefore, the point $C$  where the circles cross is equidistant from \emph{both} $A$ and $B$.

For this, we don't really need the entire circles, just the part where the arcs cross at $C$.

\begin{center} \includegraphics [scale=0.4] {PI_1c.png} \end{center}

Now use the straight edge to draw $\triangle ABC$.  Since $AC = AB$ and $BC = AB$, we know that $AC = BC$.  The triangle is equilateral.

$\square$

The proof doesn't stand on its own.  We used one definition (D) and a common notion (CN).

$\circ$ \ D I.15  all radii of a circle are equal.

$\circ$ \ CN I.1  things which equal the same thing also equal one another.

If we look again at the figure, and label the other point where the circles cross as $D$:
\begin{center} \includegraphics [scale=0.3] {PI_1d.png} \end{center}

\subsection*{Euclid I.9:  bisection of an angle}

\label{sec:Euclid_I_9}

To bisect a given angle.

\begin{center} \includegraphics [scale=0.4] {PI_9a.png} \end{center}

As radii of a circle on center $O$, we first find points $A$ and $B$ equidistant from $O$ (left panel).  Let that distance be $r$.

As radii of circles on the centers $A$ and $B$ that pass through $O$ (so the radius is equal to $r$), we find $C$ equidistant from $A$ and $B$ (middle panel), with radius also equal to $r$.

Thus, $OA = OB = AC = BC$ (right panel).  

So $\triangle OAC \cong \triangle OBC$ by SSS.

Therefore $\angle BOC$ is equal to $\angle AOC$ and the given angle is bisected.

$\square$

\subsection*{perpendicular lines}

When constructing a line segment perpendicular to another line segment, there are three common situations (see the figure below):

- We want the bisector to pass through a given point $P$ on the line.

- We know points $Q$ and $R$ on the line and want the bisector to pass halfway between them

- We want the bisector to pass through a point $S$ not on the line

\begin{center} \includegraphics [scale=0.4] {perp8.png} \end{center}

We solve the second case and then show how the other two can be adapted to it.

\subsection*{Euclid I.10:  perpendicular bisector}

\label{sec:Euclid_I_10}

Simply construct two circles of equal radius, one centered at $Q$ and the other at $R$.  It's easiest to choose the radius to be equal to the length $QR$ (left panel, below).

\begin{center} 
\includegraphics [scale=0.3] {perp9.png} 
\includegraphics [scale=0.3] {perp10.png} 
\end{center}

The point $S$ is on both circles, hence it is a radius of both, and therefore equidistant from $Q$ and $R$.  $QS = SR = QR$.  

The three points form an equilateral triangle, $\triangle QRS$ (Euclid \hyperref[sec:Euclid_I_1]{\textbf{Euclid I.1}}).

The point $T$ below the line segment has the same property, and $\triangle QRS$ is congruent to $\triangle QRT$ by SSS.

Furthermore, we claim that the line segment $ST$ is perpendicular to $QR$ and the point $P$ where $ST$ intersects $QR$ is such that $QP = PR$, and $SP = PT$.

Euclid's proof is very simple. 

\emph{Proof}.

$\triangle QRS$ is equilateral.  Bisect $\angle QSR$ (Euclid I.9) by constructing $\triangle QRT$ on QR and join $ST$.  Since the angle at $S$ is bisected and $SQ = SR$, by our work on the \hyperref[sec:isosceles_triangle_theorem]{\textbf{isosceles triangle theorem}}, the angles at $P$ are all right angles.

$\square$

Here is another way to proceed.

\emph{Proof}.

We are given that $\triangle QRS \cong \triangle QRT$ by SSS (see above).  So both angles at $Q$ (right panel) are equal, and then $\triangle QPS \cong \triangle QPT$ by SAS.

Therefore the base angles at $P$ are equal and thus right angles, and therefore all the angles at $P$ are right angles.

$\triangle PRS \cong \triangle QPS$ by hypotenuse-leg in a right triangle (HL).  It follows that $QP = PR$ so $SPT$ bisects $QR$ and also forms a right angle where it intersects at $P$.

\begin{center} 
\includegraphics [scale=0.3] {perp9.png} 
\includegraphics [scale=0.3] {perp10.png} 
\end{center}

All four smaller triangles are congruent by SAS or SSS or HL.

$\square$

\subsection*{Euclid I.11:  bisector through a given point on the line}

\label{sec:Euclid_I_11}

Suppose we know a point $P$ on the line and wish to construct the vertical line through $P$.  

Use the compass to mark off points $Q$ and $R$ on both sides of $P$, equidistant from it.  This can be done by drawing a circle with center at $P$.

\begin{center} \includegraphics [scale=0.4] {perp_7.png} \end{center}

Now, simply proceed as above.

\subsection*{Euclid I.12:  bisector through a given point not on the line}

\label{sec:Euclid_I_12}

Alternatively, suppose we know the line and the point $U$ but not $P$, and we wish to construct a vertical through the line that also passes through $U$.  

Find $Q$ and $R$ on the line an equal distance from $U$ ($QU$ = $RU$), as radii of a circle centered at $U$ (left panel, below).  Their exact position is unimportant.  

\begin{center} \includegraphics [scale=0.35] {perp11.png} \end{center}

Now repeat the previous construction, using $Q$ and $R$.  The line segment $ST$ passes through $U$, as required.

\emph{Proof}.  (Sketch).  

Since $QU = RU$, $\triangle QUR$ is isosceles.  Therefore the base angles are equal.  In an isosceles triangle, the top vertex lies on the perpendicular bisector of the base  (see the chapter on isosceles triangles.

Alternatively, find a point below the line equidistant from $Q$ and $R$.  Proceed as in the proof above.

\subsection*{bisector is the altitude of an isosceles triangle}

Suppose we know two points $Q$ and $R$.  We find the point $P$ equidistant between them and construct the perpendicular bisector $PS$.  Then the two sides $SQ$ and $SR$ have equal length.  Triangle $\triangle SQR$ is isosceles.

\begin{center} \includegraphics [scale=0.45] {perp_3.png} \end{center}

We proved this in the chapter on isosceles triangles.

It is true for \emph{any} point on the line drawn through $S$ and $P$.  For example, $TQ = TR$ in the figure above.


\end{document}