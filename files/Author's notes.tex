\documentclass[11pt, oneside]{article} 
\usepackage{geometry}
\geometry{letterpaper} 
\usepackage{graphicx}
	
\usepackage{amssymb}
\usepackage{amsmath}
\usepackage{parskip}
\usepackage{color}
\usepackage{hyperref}

\graphicspath{{/Users/telliott/Dropbox/Github-math/figures/}}
% \begin{center} \includegraphics [scale=0.4] {gauss3.png} \end{center}

\title{Introduction}
\date{}

\begin{document}
\maketitle
\Large

%[my-super-duper-separator]

\label{sec:additional_notes}

A central feature of this book is the relentless use of proof.  I emphasize the key insight for each, and have tried to make the proofs simple and as easy to follow as possible.

This volume is distinguished from most other texts, since they maintain that a proper proof should be watertight, with each step carefully justified and following closely from the one before.  I don't deny that rigor has its proper place in math education, but I also think that this rigidity obscures the core insights.  Our purpose here is to view beauty clearly.

We prefer instead to be like the famous mountaineer Ueli Steck.  Reach the summit quickly and emphasize the key steps.  You should be able to fill in the details if there are loose ends.

Multiple proofs for important theorems are sometimes given, because proof is our stock in trade, and different approaches shed light on how proofs may be found and developed.

Trigonometry comes early.  It is often held back for another course, but it is extremely useful to abstract the notion of ratios of sides, so we do it in the first part of the book.  

Another distinguishing feature is a set of simple proofs based on scaling of triangles.  This happens for the Pythagorean theorem and for Ptolemy's theorem, as well as the sum of angles theorems and then later, a fairly sophisticated theorem of Euler's.

Recently I came across a fantastic book by Acheson, called \emph{The Wonder Book of Geometry}.  I helped myself to some of his examples, and now have more than a dozen.  Please go find Acheson, and buy it.  It's truly magical.  In fact, all of his books are wonderful!

A saying attributed to Manaechmus, speaking to Alexander the Great, is that ``there is no royal road to geometry".  Others write that this was actually Euclid, speaking to Ptolemy I of Egypt.  Since the two sources lived some 700 years after the fact, it is difficult to know.

Practically, this means that in learning mathematics you must follow the argument with pencil and paper and work out each step yourself, to your own satisfaction.  That is the only way of really learning, and at heart, a principal reason why I wrote this book.  

Having read a chapter, see if you can prove the theorems yourself, without looking at the text.

There are a few problems listed in the later chapters, perhaps thirty or more altogether.  Most of them have worked out solutions.  It is highly recommended that you attempt each problem yourself before reading my answer.  Since the crucial point is often to draw an inspired diagram, you must stop reading as soon as the problem is stated!

\end{document}