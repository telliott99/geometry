\documentclass[11pt, oneside]{article} 
\usepackage{geometry}
\geometry{letterpaper} 
\usepackage{graphicx}
	
\usepackage{amssymb}
\usepackage{amsmath}
\usepackage{parskip}
\usepackage{color}
\usepackage{hyperref}

\graphicspath{{/Users/telliott/Github-math/figures/}}
% \begin{center} \includegraphics [scale=0.4] {gauss3.png} \end{center}

\title{Broken Chord}
\date{}

\begin{document}
\maketitle
\Large

%[my-super-duper-separator]

The theorem of the "broken chord" is ascribed to Archimedes, although his original work --- the \emph{Book of Circles} --- has been lost.  It was analyzed in proofs collected by the Arabic mathematician Al Biruni in his \emph{Book on the Derivation of Chords in a Circle}.  The theorem was not simply of academic interest, but related to the construction of tables of chords (see the later chapter on the \hyperref[sec:xyz]{\textbf{Almagest}}).

Here is the general setup:
\begin{center} \includegraphics [scale=0.4] {broken_chord2.png} \end{center}

Let $A$ and $C$ be any two points on a circle, and let $M$ be equidistant from both, so that arc $AM$ is equal to arc $MC$.  Let $B$ be another point on the circle, lying between $A$ and $M$, so $AB < BC$.

Drop the perpendicular from $M$ to $F$ on $BC$.  The claim of the theorem is that $AB + BF = FC$.  

In this chapter we will look at several proofs including those ascribed to Archimedes.  My original source for this problem was  

\url{https://www.uni-miskolc.hu/~matsefi/HMTM_2020/papers/HMTM_2020_Drakaki_Broken_chord.pdf}

There, it is said that the book contains 22 proofs of the theorem, including 3 different ones due to Archimedes.

Later I found a link to Al Biruni's book, in Arabic:

\url{https://tile.loc.gov/storage-services/service/gdc/gdcwdl/wd/l_/07/46/9/wdl_07469/wdl_07469.pdf}

Using some of the diagrams in that book I worked out some more proofs.  Then, even more recently, I came across a translation, written in German by a historian named Heinrich Suter:

\url{http://www.jphogendijk.nl/biruni/Suter-Chords.pdf}

I have translated a part of that book into English.  The scanned text is copyable, and Google Translate does a decent job, but there are a lot of errors in the character recognition.  I'm not complaining --- I think it's amazing that it works at all.

I count 23 proofs (labeled $a$ through $x$, with no $j$), and four of those are Suter's own proofs, which gives a total of 19.  (Although some of these contain multiple approaches with the same diagram).  We'll return to Suter's book later.  

This topic is a great one for ``elementary" geometry because the proofs are fairly easy and there are so many of them.

Before starting, let's recall the important consequence (Euclid III.21) of the inscribed angle theorem (Euclid III.20).

By definition, the central angle sweeping out a given arc is equal in measure to the length of the arc.  Any peripheral angle subtended by the same arc is one-half that central angle.  We will not repeat the proof here but just show the figure to remind us.

\begin{center} \includegraphics [scale=0.15] {inscribed angles.png} \end{center}

$\angle PRQ = \angle PSQ$.  Angles that lie on the same arc or are subtended by the same chord in the same circle, are equal.  

We will also need the theorem that, in a given circle, equal arcs correspond to equal chords, and vice-versa, from \hyperref[sec:equal_arcs_equal_chords]{\textbf{here}}.

And we will also use several times the theorem that any point on the perpendicular bisector of a segment forms an isosceles triangle when connected with the endpoints of the segment.

\subsection*{First proof}

The first and second proofs are ones actually attributed to Archimedes.

\emph{Proof}.

Referring to the diagram below, we extend the chord $BC$ to $D$ so that $CF = DF$.
\begin{center} \includegraphics [scale=0.4] {broken_chord13.png} \end{center}

Since $CF = DF$ and $MF \perp DC$, we have that $\triangle CMF \cong \triangle DMF$.

Thus, $CM = DM$ and the angles at $C$ and $D$ are equal.  Now connect a few more points.
\begin{center} \includegraphics [scale=0.4] {broken_chord14.png} \end{center}
$\angle BAM$ (third magenta dot) is equal to $\angle C$ by the \hyperref[sec:equal_angles_same_arc]{\textbf{inscribed angle theorem}}.

We know that $AM = CM$ because the corresponding arcs are given as equal, and $CM = DM$, so $AM = DM$, which means that $\triangle ADM$ is isosceles.

As the base angles of an isosceles triangle $\angle ADM = \angle DAM$.  The magenta parts are equal, so the difference of angles is equal as well.  Namely, $\angle BDA = \angle DAB$ and $BD = AB$ by the converse of the isosceles triangle theorem.

But $BD + BF$ is given as equal to $CF$, so substituting, $AB + BF = CF$.

$\square$

\subsection*{second proof}

\begin{center} \includegraphics [scale=0.4] {broken_chord18a.png} \end{center}
Given that $M$ is the midpoint of the arc $ABC$ and $MF \perp BC$.

Find $E$ such that $BF = FE$ and also find point $D$ on the circle such that $BM = MD$.  

We will show that $\triangle MEC \cong \triangle MDC$.

From $BF = FE$ and $MF \perp BC$, we can show that $\triangle BFM \cong \triangle EFM$, so $\triangle MBE$ is isosceles and has the angles marked with black dots equal.

Next, the angles marked with red dots are equal, because the arcs they are subtended by were given as equal, $BM = MD$.

Now, $\angle FEM$ (black dot) is equal to $\angle EMC + \angle ECM$, red plus magenta, by applying the external angle theorem to $\triangle MEC$.

It remains to show that $\angle DMC$, marked with blue, is actually equal to magenta.

We obtain this by looking at the arcs.  arc $MDC$ subtends $\angle MBC$, black dot, but this is the sum of the arcs for $\angle DMC$ plus $\angle DCM$.  Since we already have that $\angle DCM$ is red, the result follows.

\begin{center} \includegraphics [scale=0.4] {broken_chord18b.png} \end{center}

Therefore $\triangle MEC \cong \triangle MDC$ by ASA.  So $CD = EC$.

$AB$ is equal to $CD$, because when added to equals ($BM = MD$), they combine to give equals:
\[ AB + BM =  AM = MC = MD + CD \]

It follows that $EC = CD = AB$.

Since $BF = EF$, adding equals to equals gives
\[ AB + BF = EC + FE = FC \]

$\square$

\subsection*{third proof}

This elegant proof is ascribed to Gregg Patruno, a student at Stuyvesant High School in New York (1980).  I found it in

\url{https://www.researchgate.net/publication/341579803_FROM_THE_THEOREM_OF_THE_BROKEN_CHORD_TO_THE_BEGINNING_OF_TRIGONOMETRY}

[Note:  Drakaki attributes it to Patruno, but it is also in Al Biruni's book, according to Suter's translation.]

\begin{center} \includegraphics [scale=0.45] {broken_chord3.png} \end{center}

\emph{Proof}.

Draw $E$ such that $EC = AB$.  We will prove that $BF = FE$ so $AB + BF = FC$.

We have that $MA = MC$ as two chords cutting out equal arcs in a circle.  We also have that $\angle A$ and $\angle C$ both are subtended by the same arc $BM$, so they are equal.  $EC = AB$ was given.  

Therefore, $\triangle ABM \cong \triangle CEM$ by SAS.

As corresponding sides in congruent triangles, $MB = ME$.  Therefore $\triangle MBE$ is isosceles.

The altitude $MF$ cuts isosceles $\triangle MBE$ into two congruent right triangles, which means that the bases are equal.  $BF = FE$.

We drew $E$ such that 
\[ AB = EC \]
Adding equals to equals 
\[ AB + BF = FE + EC = FC \] 

$\square$

\subsection*{fourth proof}

Here is a another simple proof of the broken chord theorem.  This one is said to be in Al-Biruni's text and is attributed to El-Sidjzi (972), but may actually have been known to Apollonius.

\url{https://www.researchgate.net/publication/341579803_FROM_THE_THEOREM_OF_THE_BROKEN_CHORD_TO_THE_BEGINNING_OF_TRIGONOMETRY}

This figure from Al Biruni (below) suggests a neat line of attack.
\begin{center}
\includegraphics [scale=0.35] {Al_Biruni_5.png}
\end{center}

It is worthwhile to recall a figure from our work on parallel chords in a circle.  
\begin{center} \includegraphics [scale=0.5] {parallel_chords.png} \end{center}

Draw $AB \parallel PQ$.  Then draw $AQ$.  The angle at $Q$ is equal to the angle at $A$, by alternate interior angles.  Therefore arc $AP \cong $ arc $BQ$ by the inscribed angle theorem and then $AP = BQ$ as chords of equal arcs.

If we drop perpendiculars from $A$ and $B$ to $PQ$, forming a rectangle, the two small triangles flanking the rectangle are congruent, by HL in a right triangle.

In other words, for any rectangle whose vertices on one side are on a circle, the opposing side is centered in the circle.  The converse is also true.  If $BF = GC$ and $MF, HG \perp BC$, then $MHFG$ is a rectangle.

\begin{center} \includegraphics [scale=0.4] {broken_chord9.png} \end{center}

\emph{Proof} (of the broken chord theorem).

As before, $M$ is the midpoint, with arc $AM$ equal to arc $MC$.  $MF \perp BC$.

This time we divide up $FC$ in a different way.  Either we have $BF = CG$, and then $MFGH$ is a rectangle, or we have the converse, and it follows that $BF = GC$.  

\begin{center} \includegraphics [scale=0.4] {broken_chord9.png} \end{center}

From the rectangle, we also know that $MH = FG$  So then from the property of the median
\[ AB + BM = MH + CH \]
subtracting equals and using the rectangle
\[ AB = MH = FG \]
Finally, adding equals to equals
\[ AB + BF = FG + GC = FC \]

$\square$

\subsection*{fifth proof}

Here is another diagram from the book in Arabic that I can't read.  It suggests another proof and I couldn't resist working something out.

\begin{center} \includegraphics [scale=0.35] {Al_Biruni_3.png} \end{center}

Let's draw this using our standard notation.

\begin{center} \includegraphics [scale=0.4] {broken_chord19.png} \end{center}
We find the point $E$ as before, by requiring that $BF = FE$.  Simply extend $ME$ to meet the circle at $G$ and connect $G$ to both $A$ and $C$.  It looks like $AG$ might be parallel to $BC$.

The angles with black dots are equal for the following reasons:  (i) the top two are the base angles in an isosceles triangle, as before;  (ii) the other one at $E$ is a vertical angle, and (iii) the one at $G$ is an inscribed angle corresponding to the same arc as the black dotted angle at $B$.

So we see that $\triangle CEG$ is also isosceles, which means that $EC = CG$.
\begin{center} \includegraphics [scale=0.4] {broken_chord20.png} \end{center}
The two triangles $\triangle MBE$ and $\triangle CEG$ are similar, which accounts for the magenta dots.  I have added a second black dot at $G$.  The justification is that it is subtended by the arc $AM$, given as equal to $MC$, which subtends its twin at $G$.

But the sum of the three angles at $C$ and $G$ is equal to two right angles, from comparison to $\triangle MBE$, which means that $BC \parallel AG$.

The final step is to show that parallel chords in a circle cut off equal arcs.  

The figure from Al Biruni (below) reminds us that we just did this in the previous proof:

\begin{center}
\includegraphics [scale=0.35] {Al_Biruni_5.png}
\end{center}

In our standard diagram, draw one of the diagonals of the quadrilateral such as $AC$.  Because two sides $BC$ and $AG$, the alternate interior angles are equal.  But that means the intercepted arcs and corresponding chords are equal.

$\square$

\subsection*{sixth proof}

We switch to Suter's notation --- this proof is Suter s(ii).
\begin{center} \includegraphics [scale=0.3] {Suter14.png} \end{center}

Suppose we have $AZ = GB$, and now we draw $AH = MG$.  We also have $\angle A = \angle G$.  It follows that $\triangle DZA \cong \triangle DBG$ by SAS and $\triangle AZH \cong \triangle BMG$, but the difference is also congruent, $\triangle DZH \cong \triangle DMB$.  This latter equality gives side $DZ = DB$, so $\triangle DBZ$ is isoceles.

We are going to play with the areas of triangles and relate them to the products of base $\cdot$ height.  I made a new diagram showing what we know so far.
\begin{center} \includegraphics [scale=0.3] {Suter14b.png} \end{center}
Also, we know that $1 + 3 = 4$.  (I didn't plan that, but it's nice.  These are just labels).  So then, recall what we want to prove:
\[ AE = BE + BG = BE + AZ \]
Multiply both sides by the vertical, $DE$
\[ DE \cdot AE = DE \cdot BE + DE \cdot AZ \]
Recasting these as triangular areas, e.g. $\triangle DZA = (1,2)$, we can write
\[ 2 \cdot (1,2,4) = 2 \cdot (1,3) + 2 \cdot (1,2) = 2 \cdot (4) + 2 \cdot (1,2) \]  
$\square$

\subsection*{seventh proof (Suter x)}

Again, we use his notation.
We are given the isosceles $\triangle DBH$, so $BE$ = $EH$ and $DB = DH$.

\begin{center} \includegraphics [scale=0.3] {Suter19.png} \end{center}

We will show that $\angle ZBD = \angle DHE$.  $\angle ZBD$ is external to $\triangle DBG$, so it stands on the arc $GBD$ as the sum of the interior angles' arcs.  $\angle DBE$ stands on the equal arc $AD$ and $\angle DHE$ is equal, so $\angle ZBD = \angle DHE$.  It follows that their supplementary angles are equal:  $\angle DBG = \angle DHA$.

Therefore, we have $\triangle DBG \cong \triangle DHA$ by SAS which means that $BG = AH$, and the rest follows easily. 

 $\square$
 
\subsection*{eighth proof}

Again from Suter's German translation of Al Biruni's work

\url{http://www.jphogendijk.nl/biruni/Suter-Chords.pdf}

(Suter, section $\gamma$ , Fig. 23, p. 28).

\begin{center} \includegraphics [scale=0.25] {broken_chord_Suter.png} \end{center}

From the drawing, I imagine that the construction in the text says to extend $BG$ to where the perpendicular from $D$ cuts it at $H$ and then go further to $Z$ such that $BH = HZ$.  

First, let's redraw the figure.  At this point, I am going to switch to use Suter's notation.  (Perhaps someday I will redraw all the previous figures in the same fashion, but not yet).
\begin{center} \includegraphics [scale=0.4] {broken_chord25.png} \end{center}

So now we extend $GB$, drawn the perpendicular $DH$ and have $Z$ such that $ZH = HB$.

The two parts of the ``broken" chord are $GB$ and $AB$, and the median point is $D$.  The perpendicular from $D$ down to $BC$ terminates at $E$.

Suppose we can prove that $\triangle ABD \cong \triangle GZD$.  Then $GB + BH = AE$ and if we can also prove that $BE = BH$ we are done.

To show congruence, we will prove that $\angle Z = \angle DBA$.  Then, since $\angle A = \angle G$ by the inscribed angle theorem, and $DG = DA$ by the property of the median, we will have ASA and thus congruence.

\begin{center} \includegraphics [scale=0.4] {broken_chord25.png} \end{center}
It is easy to show that $\triangle BDZ$ is isosceles, so $\angle Z = \angle DBH$.  The key point is that the supplementary angle $\angle DBG$ cuts out all of the circle \emph{except} $GD$.  Hence $\angle DBH$ and by extension, $\angle Z$, cuts out arc $GM$.  Therefore, $\angle Z$ is equal to $\angle DBM$, which cuts out arc $DA$, equal to arc $GD$. 

We have ASA and thus congruence.  What about the last part, $BE$ and $BH$?  

A simple approach is to say:  these are altitudes from the corresponding vertex in two congruent triangles.  They form two pairs of congruent small triangles, by hypotenuse-leg in a right triangle.  Therefore, the altitudes are equal.

$\square$

\subsection*{ninth proof}
We'll do one last proof from Suter (section n, Fig 9, pp 18-19).
\begin{center} \includegraphics [scale=0.4] {broken_chord26.png} \end{center}

We start by extending the perpendicular $DE$ to $Z$ and draw $AG$, plus the diameter of the circle from $D$ through $HTK$.  The angle at $Z$ is a right angle, by Thales' circle theorem.  Finally, draw $MK$ parallel to $DEZ$.

Now, the crucial step is that $DHTK$ is perpendicular to $AG$.  The reason is that the arc $GD$ is equal to arc $AD$, since $D$ is the median.  Therefore $\triangle ADG$ is isosceles, so its perpendicular bisector splits the base in half and is also a diameter of the circle containing the three points $A$,$G$ and $D$, i.e. $DHTK$ is the perpendicular bisector of $AG$.

Clearly, because of the two right angles and parallel side opposite, $EMKZ$ is a rectangle, which means that $BE = MA$ and $EM = ZK$.

We will prove that that $\angle A = \angle D$.  We have that $\angle DEH$ is right, and so is $\angle ATH$, by vertical angles.  Therefore $\triangle DEH \sim \triangle ATH$.  It follows that $\angle A = \angle D$.  Thus,  $ZK$ is equal to $GB$, so $ZK = GB = EM$, and the result follows easily.

$\square$

\subsection*{on the value of SSA}

\label{sec:use_of_SSA}

Something interesting happens with the third proof if you approach the premises slightly differently.
\begin{center} \includegraphics [scale=0.4] {broken_chord4a.png} \end{center}
We draw $AM$ as before, but we forget to set $CE = AB$ and instead put $BF = EF$, as in some other proofs.  Then what happens?

We have $SAS$ in the small right triangles so $\triangle BFM \cong \triangle EFM$, which means $BM = ME$.  We have $MA = MC$ as before and $\angle A = \angle C$.

We are tempted to compare $\triangle ABM$ with $\triangle CEM$.  What we have is SSA (or ASS), which --- it's been drilled into our heads --- is \emph{not enough}, unless there is a right angle, in which case we call it hypotenuse-leg in a right triangle (HL).

But let's take a closer look.  In the figure below, the angle that we know is marked with a black dot, and the two sides that we know are $b$ and $a$.
\begin{center} \includegraphics [scale=0.4] {broken_chord4d.png} \end{center}
With SSA there are the following possibilities.  

$\circ \ $ $a$ may be so short that it can't reach the base and there is no triangle.

$\circ \ $ $a$ is just long enough that it forms a right angle with the base.

$\circ \ $ $a$ is longer than the right angle case, in which case there are two possible triangles.

$\circ \ $ $a$ is equal to $b$, in which case there is only one possible triangle, an isosceles one.

The usual example where we might want to get some information from SSA is the third one.
\begin{center} \includegraphics [scale=0.4] {broken_chord4d.png} \end{center}
We note that $a$ is the side opposite the angle we know, and we are concerned with the angle it forms with the side we don't know.

In one, the angle $\phi$ with the undetermined base is acute, and in the other it is obtuse.  These angles are supplementary.

We can see both of them in the figure!
\begin{center} \includegraphics [scale=0.4] {broken_chord4a.png} \end{center}

Clearly, and it is easy to prove, in $\triangle ABM$ and $\triangle CEM$ the angle is obtuse, from which we conclude that $\triangle ABM \cong \triangle CEM$.  Therefore, $CE = AB$ and adding equals to equals, we are done.

$\square$

\begin{center} \includegraphics [scale=0.45] {Al_Biruni_2.png} \end{center}

This proof is also in Al Biruni's book on page 15.  

From reading Suter, the way in which Archimedes handles the problem is stellar!

\begin{center} \includegraphics [scale=0.75] {Suter2a.png} \end{center}
He says (re-phrasing):  $GD = AD$ by the property of the median, $BD$ is shared, and $\angle G = \angle A$, so we have SSA.

Now, $Z$ has been drawn such that $ZE = AE$ which means $ZD = AD$, so we have actually three triangles with SSA.

Archimedes finds another angle.  He says that $\angle GBD$ corresponds to everything except arc $GD$, i.e. $DAHG$.  The angle supplementary to $\angle ZBD$ is subtended by arc $AD$, so $\angle ZBD$ itself is subtended by everything except arc $AD$, i.e. $AHGB$.  

The missing parts are equal, so the angles are equal!  Since we know two angles, we know three.  Therefore, $\triangle ZBD \cong \triangle GBD$ by ASA.

\subsection*{more}

There is one more proof of the broken chord theorem that I know of.  (Remember the source claimed 22 proofs!)

It is on the web attributed to someone named Bùi Quang Tuån.  There is another proof from the same source of the \hyperref[sec:Pthm_Tuan]{\textbf{Pythagorean theorem}} that I like even better. 

\url{https://www.cut-the-knot.org/pythagoras/BrokenChordPythagoras.shtml}

Google also turns up a blog, but no biographical info.

\url{https://artofproblemsolving.com/community/c1598}  

This proof is based on a rectangle, and I leave it to you to see how it relates to the fourth proof, above.  I've written about it elsewhere. 

Note:  some additional material is here:

\url{https://www.cut-the-knot.org/triangle/BrokenChordmpdlc.shtml}

\subsection*{\emph{Star of David} proof of the Pythagorean theorem}

\label{sec:star_of_david}

\url{https://www.cut-the-knot.org/pythagoras/PythStarOfDavid.shtml}

\begin{center} \includegraphics [scale=0.35] {pyth21.png} \end{center}
Draw two congruent mirror-image right triangles in a circle, oriented so that $EF \perp AB$ (and $BC \perp DE$).  

Note that $AB$ and $DE$ both pass through $O$, the center of the circle, because any right triangle inscribed in a circle has its hypotenuse as a diameter, by the converse of Thales' circle theorem.

$OA$ and $OD$ are perpendicular by construction and diagonals, so they are perpendicular bisectors.  Thus
\[ AE = AF = DC = DB \]

Arc $EC$ plus arc $CD$ is equal to $180^{\circ}$.

So it is equal to the arc $EB$, which added to arc $DB$ gives 180.

Then $E$ is the median point between $B$ and $C$ and the perpendicular dropped from $E$ which meets $AB$ in a right angle, cuts $AB$ so
\[ AP + AC = PB \]
\[ AC = PB - AP \]
by the theorem of the broken chord.  The two pieces of $AB$ are
\[ AB = AP + PB \]

Putting this together, we have:
\[ AB + AC = 2PB \]
and
\[ AB - AC = 2AP \]
Hence
\[ PB = \frac{1}{2} (AB + AC), \ \ \ \ \ \ AP = \frac{1}{2} (AB - AC) \]

\begin{center} \includegraphics [scale=0.35] {pyth21.png} \end{center}

By the theorem of crossed chords
\[  \frac{AB + AC}{2} \cdot \frac{AB - AC}{2} = ( \frac{EF}{2} )^2 \]
\[  AB^2 - AC^2 = EF^2 = BC^2 \]

The result follows.

$\square$

\end{document}