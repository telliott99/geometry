\documentclass[11pt, oneside]{article} 
\usepackage{geometry}
\geometry{letterpaper} 
\usepackage{graphicx}
	
\usepackage{amssymb}
\usepackage{amsmath}
\usepackage{parskip}
\usepackage{color}
\usepackage{hyperref}

\graphicspath{{/Users/telliott/Github-Math/figures/}}

\title{Isosceles triangles}
\date{}

\begin{document}
\maketitle
\Large

%[my-super-duper-separator]

Triangles are often classified on the basis of the largest angle they contain:  acute, right, or obtuse.  
\begin{center} \includegraphics [scale=0.4] {tri_types.png} \end{center}

The acute triangle (left) has all three angles smaller than a right angle.  The right triangle, naturally, has one right angle.  We'll say a lot more about right triangles later.

An obtuse triangle has one angle larger than a right angle (right panel, above).

\subsection*{symmetry}

One can also talk about the situation where either two sides, or all three sides, have the same length.  An \emph{isosceles} triangle has two sides the same length, while an \emph{equilateral} triangle has all three sides the same.

The most important consequence of all three sides equal for an equilateral triangle is three-fold rotational symmetry.  Three turns of $120$ degrees, and we're back where we started.  Each of the two intermediates is identical.

\begin{center} \includegraphics [scale=0.4] {equilateral.png} \end{center}

The implication of rotational symmetry is that the three angles are also equal because there is no reason to choose one larger than any other.  

Therefore each angle of an equilateral triangle is $2/3$ of a right angle, or $60^{\circ}$, by the triangle sum theorem (\hyperref[sec:triangle_sum_theorem]{\textbf{ref}}).

It is also true that if all three angles are equal, then the triangle is equilateral (three sides equal).  We will show how to prove this later.

\subsection*{notation}

To repeat something we said previously, the Greeks, including Euclid, always label points with letters, while line segments are referred to by the endpoints.  Angles and triangles are denoted by the line segments from which they are composed, as in $\angle ABC$, and triangles by their vertices:  $\triangle ABC$.

\begin{center} \includegraphics [scale=0.4] {triangle7.png} \end{center}

The Greek notation (left panel) is problematic because it is more complicated than need be. I have found myself tracing out angles from three points, often again and again.

Quite often, we will label the side opposite a vertex with a lower case letter:  side $a$ lies opposite $\angle A$.  We may use letters like $\theta$ and $\phi$ for angles, or even $s$ and $t$.

Sometimes, we will boldly dispense with labels altogether and use colored dots for equal angles and more rarely, colored bars for equal lengths.  The image below is from the web, it uses the convention of an arc drawn across equal angles.

\subsection*{theorem from Thales}

$\bullet$ \ If a triangle is isosceles (two sides equal), then the base angles are also equal.

The converse is

$\bullet$  \ If two base angles are equal, then the triangle is isosceles.

My favorite proof of both theorems about isosceles triangles is from reflective or mirror image symmetry.  

\begin{center} \includegraphics [scale=0.4] {isosceles.png} \end{center}

\emph{Proof}.

Imagine that the triangle sticks straight up from the plane like one of those standing stones at Stonehenge.  Think about walking around the triangle and looking from behind, then it would look exactly the same.  We would say that the left side as viewed from the front is equal to the right side as also viewed from the front, because if we walk around behind the triangle the right side becomes the \emph{left} and vice-versa.

Much later than Euclid, Pappus invokes SAS on the mirror image, rather than thinking about the plane being in 3D space.

The top vertex angle is shared.  So the triangle as we look from the front is equal by SAS to the one where we look from the back.  It follows that the base angles are equal.

$\square$

\subsection*{proofs based on triangle bisection}

\label{sec:isosceles_triangle_theorem}

The forward theorem about isosceles triangles is:

$\bullet$ \ Two sides equal $\Rightarrow$ opposite angles equal.

There is a simple demonstration based on angle bisection.  [We won't call it a proof.  See the discussion below.]

\begin{center} \includegraphics [scale=0.4] {iso6b.png} \end{center}

\emph{Demonstration}.

We are given $AB = AC$ (left panel).  

Draw the bisector of the top angle $A$ (middle panel).  This construction forms equal angles at the top, marked with red dots.  

But then, the two smaller triangles $\triangle ABD$ and $\triangle ACD$ are congruent by SAS since

$\circ$ \ \ $AB = AC$

$\circ$ \ \ $\angle BAD = \angle DAC$

$\circ$ \ \ $AD$ is shared 

We write that conclusion as $\triangle ABD \cong \triangle ACD$.

$\square$

Therefore (as corresponding parts of congruent triangles) the base angles are equal (right panel).  Other corresponding angles and sides are equal as well.

The full set of equal angles and sides is:

\begin{center} \includegraphics [scale=0.4] {iso14b.png} \end{center}

The base is bisected, plus there is a right angle where the bisected cuts the base, plus of course the central line segment $AD$, which is equal to itself.

\subsection*{converse}

\label{sec:isosceles_converse}

We will prove the converse theorem:

$\bullet$ \ Two angles equal $\Rightarrow$ opposite sides equal.

\begin{center} \includegraphics [scale=0.4] {iso7b.png} \end{center}

\emph{Proof}.

We are given that the angles marked with black dots are equal. 

We again draw the bisector of the angle $A$.  Then we have all three angles the same, and the side $AD$ is shared.

Therefore, $\triangle ABD \cong \triangle ADC$ by ASA.

Thus, $AB = AC$.  As before, there are right angles at the base, and the base is bisected.

$\square$

Euclid's proof of the isosceles triangle theorem is more complicated that what we have given above, and there is a good reason for this.  Our proof depends on the existence of the angle bisector, but we haven't actually shown how to do that.  It will turn out that our \hyperref[sec:Euclid_I_9]{\textbf{construction}} \emph{depends on} the isosceles triangle theorem.  

That's a problem because the reasoning is circular, thus invalid.  We cannot use $p$ to prove $q$ if we have previously used $q$ to prove $p$.  That proves nothing.

Nevertheless, we prefer to sidestep Euclid's proof at this point.  You can find it \hyperref[sec:Euclid_I_5]{\textbf{here}}.

\subsection*{other proofs}

We used the angle bisector at vertex $A$ for the proofs above.  But we might equally have constructed a right angle at $D$, or bisected the base.  We will just show diagrams for both proofs adapted to these methods and sketch the idea.

\begin{center} \includegraphics [scale=0.4] {iso8.png} \end{center}

Above, we are given $AB = AC$ and right angles at $D$ (and $BCD$ \emph{colinear}).  We have hypotenuse-leg in a right triangle (HL), so the two small triangles are congruent.

\begin{center} \includegraphics [scale=0.4] {iso9.png} \end{center}

Above, we are given equal angles at $B$ and $C$ and right angles at $D$.  We have three angles equal plus a shared side, so congruent triangles by AAS (or after application of the triangle sum theorem, by ASA).
 
\begin{center} \includegraphics [scale=0.4] {iso10.png} \end{center}

Above, we are given $AB = AC$ and $BD = DC$, i.e. the base is bisected.  Since $AD$ is shared we have that $\triangle ABD \cong \triangle ADC$ by SSS.

\begin{center} \includegraphics [scale=0.4] {iso11.png} \end{center}

Finally, we are given equal angles at $B$ and $C$ and we bisect the base.  We have SSA but do not know yet about the right triangle.  This presents some difficulties.  We will avoid those subtleties for now.  

\end{document}