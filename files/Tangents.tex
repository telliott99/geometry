\documentclass[11pt, oneside]{article} 
\usepackage{geometry}
\geometry{letterpaper} 
\usepackage{graphicx}
	
\usepackage{amssymb}
\usepackage{amsmath}
\usepackage{parskip}
\usepackage{color}
\usepackage{hyperref}

\graphicspath{{/Users/telliott/Github-math/figures/}}
% \begin{center} \includegraphics [scale=0.4] {gauss3.png} \end{center}

\title{Tangents}
\date{}

\begin{document}
\maketitle
\Large

%[my-super-duper-separator]

\subsection*{tangent:  perpendicular $\rightarrow$ touches one point}

\label{sec:tangent_one_point}

We show \hyperref[sec:Euclid11]{\textbf{here}} that one can construct a line perpendicular to any given line, and passing through any point whether on the line or not.

If we make the construction perpendicular to the radius (or diameter) at the point where it meets the circle, the new line is called a tangent line (from the Latin \emph{tangere}, to touch).  

$\bullet$   The tangent line, defined as perpendicular to the radius, touches the circle at a single point.

\emph{Proof.}

By definition, the tangent line at $P$ is perpendicular to the diameter or radius including $O$ and $P$.  Let $Q$ be some other point on that tangent line.  Then $\angle OPQ$ is a right angle.

\begin{center} \includegraphics [scale=0.4] {circle3.png} \end{center}

We suppose the tangent line also touches the circle at a second point $T$.

Draw the line segment $OT$.  Since $\angle OPT = \angle OPQ$ is a right angle, the line segment $OT$ is the hypotenuse in a right triangle having $OP$ as one of its sides.

But the hypotenuse is the longest side in any right triangle.  Therefore $OT > OP$.

It follows that the point $T$ cannot be on the circle, since $OT \ne OP$.  This is a contradiction.

In fact, $OT > OP$.  So $T$ must lie outside of the circle.

$\square$

\begin{center} \includegraphics [scale=0.4] {circle3.png} \end{center}

\subsection*{tangent:  touches one point $\rightarrow$ perpendicular}

\label{sec:tangent_perpendicular}

An alternative definition of the tangent is that it is a line that touches the circle at just one point.  One can use this definition to prove that the angle between the tangent and the radius is a right angle.  This is the converse of the previous theorem.

$\bullet$   The tangent line, defined as touching the circle at a single point, is perpendicular to the radius.

\emph{Proof}.

Draw the line that touches the circle at only one point $P$, and draw the radius to that point, $OP$.  

If we assume that $OP$ is not perpendicular to $QPT$, we can derive a contradiction.

Consider a succession of points moving along the line away from $P$ in either direction.  The angle formed with a line drawn through $O$ gets smaller as the points get farther from $P$.

The angle between the tangent and the radius is greater than a right angle on one side of $P$ (since the angle at $P$ is not a right angle).  On that side, move along the line until we find the point that does form a right angle with a radius of the circle.  

Let us suppose the point is $T$, in the figure above.  $OTP$ is a right angle and we are in doubt about whether $T$ lies inside or outside the circle.

\begin{center} \includegraphics [scale=0.4] {circle3.png} \end{center}

But then $OTP$ is a right angle, with the side opposite, namely, $OP$, the hypotenuse of a right triangle.  But the hypotenuse is the longest side in a right triangle, so therefore $OT < OP$, so the point $T$ is \emph{inside} the circle.

We have a line with one point on the circle, and another point inside the circle.  Any line through an interior point of a circle must cross the circle at two points, which contradicts the assumption above.  Hence the angle at $P$ is a right angle.

$\square$

\subsection*{shortest distance to the circle}

Let $A$ be any point inside a circle.  Draw the radius that passes through $A$ to point $P$ on the circle.  I claim that the length $AP$ is smaller than the distance to \emph{any} other point on the circle, such as $Q$.
\begin{center} \includegraphics [scale=0.45] {tangent3d.png} \end{center}

\emph{Proof}.

Draw the radius $OQ$, which is equal to radius $OP$.
\[ OQ = OP = OA + AP \]

By the \hyperref[sec:triangle_inequality]{\textbf{triangle inequality}}
\[OA + AQ > OQ \]

so
\[OA + AQ > OA + AP \]
\[ AQ > AP \]

$\square$

\subsection*{construction of a tangent}

\label{sec:tangent_construction}

\hyperref[sec:Thales_theorem]{\textbf{Thales theorem}} provides a way to construct the tangent to a circle that passes through any exterior point $P$ --- actually, there are two such lines.

\begin{center} \includegraphics [scale=0.4] {tangent1.png} \end{center}

Use OP as the diameter of a circle.  Draw the line segment $OP$ and divide it in half by erecting the perpendicular bisector at $Q$.  Use that point $Q$ as the center of a new circle with radius $OQ$.  The point $T$ is the intersection of the two circles.

\begin{center} \includegraphics [scale=0.4] {tangent2.png} \end{center}

By the theorem, $\angle OTP$ is a right angle, and since $OT$ is a radius of the original circle, $TP$ is the tangent to the smaller circle at the point $T$.

To construct a tangent on a circle at a given point $T$:

\begin{center} \includegraphics [scale=0.4] {tangent3.png} \end{center}

Extend $OT$ to $P$ so that $OP$ is twice the radius $OT$.  Construct the perpendicular bisector at $T$.  The bisector is also the tangent of the circle.

$\bullet$   From any external point $P$, one can draw two tangents to a circle.  These two tangents have the same length.

$\bullet$   From any external point $P$, the line to the center of the circle bisects the angle between the two tangents, as well as the angle between the radii drawn to the two points of tangency.

\begin{center} \includegraphics [scale=0.4] {tangent9.png} \end{center}

\emph{Proof.}

The angle between a tangent and the radius to the point where it touches the circle, is a right angle.  For a pair of tangent lines from a given point, there are two such points on the circle.

The base lengths are both radii, so they are equal, and there is a shared side (the dotted line segment $OP$).  

Therefore the two triangles are congruent, by hypotenuse-leg in a right triangle (HL).

The two congruent sides $a$ and $b$ are the same length.

\[ a = b \]

For any point external to a circle, two tangents to the circle can be drawn, of equal length.  The line from the point to the center of the circle bisects the angle between the two tangents.

$\square$

Two circles are tangent to each other at $A$.  In one circle draw the radius to $A$ and extend it.  This line goes through the center of the second circle.
\begin{center} \includegraphics [scale=0.45] {3pts_tangent.png} \end{center}
\emph{Proof}.

Draw the tangent to the first circle at $A$.  This line goes through a single point ($A$) on the second circle.  The line perpendicular to it (the extension through $A$) is a radius of the second circle.

$\square$

\subsection*{problem}
Prove that $\angle BAC$ is a right angle.

\emph{Solution}.

Draw the tangent line to both circles at $A$, the magenta line in the figure below. (exercise:  prove that there \emph{is} a single line that is tangent to both circles at the point where they touch).

Drop radii of both circles to their respective tangent points at $B$ and $C$ (blue).  Draw the triangles $\triangle OAB$ and $\triangle PAC$.

$OB$ and $PC$ are both $\perp BC$, since $BC$ is a tangent to both, so $OB \parallel PC$.  Therefore, the blue-dotted angles are equal, by alternate interior angles.

\begin{center} \includegraphics [scale=0.45] {3pts_tangentb.png} \end{center}

Two angles labeled $\angle s$ in $\triangle OAB$ are equal because the triangle is isosceles (with two radii for sides).  The same is true for the angles labeled $\angle t$.

The blue dotted angle is equal to $2s$ but supplementary to $2t$.  It follows that
\[ 2s + 2t = 180 \]
\[ s + t = 90 \]

Therefore, $s + t$ is equal to one right angle.  Since $\angle BAC$ adds to that to give two right angles, it is equal to one right angle.

$\square$

\subsection*{The eyeball theorem}

\label{sec:eyeball_theorem}

This problem is from Acheson's Geometry book (Fig 131).  We have two circles with tangents drawn from the center of each circle to the other one.  Let the radius of the large circle be $R$ and that of the small one be $r$.

We are to compute the length of the chords formed by the tangents as they exit the originating circle.

\begin{center} \includegraphics [scale=0.4] {eyeball1.png} \end{center}

\emph{Solution}.

Obviously, the result depends on the distance separating the two circles.  Let us call that distance $D$.

\begin{center} \includegraphics [scale=0.4] {eyeball2.png} \end{center}

The centerline (of length $D$) bisects the angle formed between the two tangents at $O$ (black lines).  \emph{Proof}:  the two right triangles containing the centers $O$ and $P$ as vertices and the black lines as hypotenuse are congruent by hypotenuse-leg in a right triangle (HL).

Let the half-angle at $O$ be $\theta$.

There are two similar right triangles with angle $\theta$.  One has side $s$ and hypotenuse $R$, while the other has side $r$ and hypotenuse $D$.  By similar triangles:
\[ \frac{s}{R} = \frac{r}{D} \]
\[ s = \frac{rR}{D} \]

This is one-half of the chord, so $2s$ is the whole.

But the result is symmetrical in $r$ and $R$.  Therefore the half-chord in the other circle has an identical length.

The chords are equal in length and parallel to each other (since they are perpendicular to the line containing $D$), so they form the sides of a rectangle.

\begin{center} \includegraphics [scale=0.4] {eyeball3.png} \end{center}

Later on we will compute the sine of the angle $\theta$ as the side opposite over the hypotenuse:

\[ \sin \theta = \frac{s}{R} = \frac{r}{D} \]
\[ s = R \sin \theta \]

The total length of the chord is twice that for a central angle of $2 \theta$, so $2R \sin 2 \theta$.  

This will be convenient because it will turn out that an inscribed angle, with the chord as its opposite side, will have a measure of one-half the central angle so its chord length will be $2R \sin \theta$.

\subsection*{penny-farthing problem problem}

Here is a problem with a fascinating history (see Acheson).  Find an expression for $D$ in terms of $a$ and $b$.

\begin{center} \includegraphics [scale=0.5] {tangent10.png} \end{center}

Its solution is very easy, so I will not give it here but I encourage you to try.  

\subsection*{problem}

Harvard 1899 exam:
\begin{center} \includegraphics [scale=0.5] {Harvard1899_4.png}  \end{center}

\emph{Solution}.

The tangent lines are perpendicular to a radius drawn to the point of tangency.  As a result, two corresponding radii meet at the point of tangency and the two corresponding centers are co-linear with the point of tangency.

\begin{center} \includegraphics [scale=0.4] {Harvard1899_4p.png}  \end{center}

Therefore, we have a triangle with sides as shown. The tangent lines are perpendicular to the sides of the triangle, but will not, in general, pass through a vertex or center of the third circle, for any pair of circles and their tangent line.

But for each vertex of the triangle, the bisector of the angle gives congruent triangles, in which one of the sides is a blue dotted line, forming a right angle with the radius.  The hypotenuse of one such pair is shown in magenta.

The two triangles that share the magenta line as hypotenuse are congruent by hypotenuse-leg in a right triangle (HL), and therefore the two blue dotted lines meeting in the center have equal length.  Now do the same for the circle with radius $r$ and then for the circle with radius $\rho$.

The blue-dotted lines are thus all equal.  So they can be used to draw a circle that just touches the sides of the triangle.  That circle is called the incircle of the triangle.

The point where these tangents meet is the incenter of the triangle.  Since the incenter exists, the three angle bisectors are concurrent, the point where the tangents meet is the same and it also exists.

$\square$

\subsection*{problem}
This is a problem from Paul Yiu.  We have a semicircle inside a square with one side of the square as the diameter.  The tangent is drawn as shown.
\begin{center} \includegraphics [scale=0.4] {pyth22.png}  \end{center}
Prove that the triangle is a $3-4-5$ right triangle.

Scale the square so that the side has length $4$.  We will use the property that the two tangents to any circle from an exterior point are equal.  Thus, the distance from the lower left to the point of tangency is $4$.  Let the other short tangents have length $x$.

Then Pythagoras says that:

\[ 4^2 + (4-x)^2 = (4+x)^2 \]
\[ 4^2 + 4^2 - 8x + x^2 = 4^2 + 8x + x^2 \]
\[ 16x = 4^2, \ \ \ \ \ \ \  x = 1 \]
So the sides are indeed $3-4-5$.

\subsection*{one more}
To prove:  $PT = RS$.  (Next figure).
All the paired short segments are equal ($u$ flanking $P$, equal to $PT$, etc.) because they are paired tangents from the same point.  Our interest is the relationship between $u,v,x$ and $y$.

It is also easy to show that the crossed segments are equal.  One is $TS$ and one crosses $TS$ but isn't labeled.  They are equal because the components are equal, being tangents from the same point to the two circles.  Let this length $ST$ be called $\Delta$.

\begin{center} \includegraphics [scale=0.35] {tangent13.png} \end{center}
As tangents from a single point to the same circle.
\[ RT = \Delta + y = B + x \]
As well
\[ B + y = \Delta + x \]
\[ B - \Delta = y - x = x - y \]
Thus
\[ x = y \]
Exactly the same argument using $A$, $u$ and $v$ gives
\[ A + v = \Delta + u \]
\[ \Delta + v = A + u \]
\[ A - \Delta = v - u = u - v \]
\[ u = v \]
Imagine that we construct tangents from the same point off-screen to the right, the first part of each is equal (tangent to the circle with center $B$) and so is the whole (tangent to the other circle), meaning that differences are also equal, namely
\[ u + A + v = x + B + y  \]
The left-hand side is (looking back)
\[ A + u + v = \Delta + 2u \]
while the right-hand side is
\[ x + y + B = \Delta + 2y \]
\begin{center} \includegraphics [scale=0.35] {tangent13.png} \end{center}
So, finally $u = y$ or $PT = SR$.
All the small segments are equal:  $u=v=x=y$, and also $A = B = \Delta$.  



\end{document}